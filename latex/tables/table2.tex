\begin{table*}[t!]
\centering
\begin{tabular}{|l|}
\multicolumn{1}{c}{\textbf{Stability Mechanisms}}  \\ \hline

\multicolumn{1}{|p{\textwidth}|}{\textbf{Directly Backed and Redeemable.}\newline \footnotesize Alice is a trusted third party and uses Ethereum to instantiate a decentralized application (DApp) which issues 1000 AliceCoins as standard tokens (\eg ERC20). She asks \$1 USD for 1 AliceCoin and promises to redeem any AliceCoin for \$1 USD. If Bob buys 10 AliceCoins for \$10 USD, Alice deposits the \$10 USD in a bank account. Any time Alice receives a buy order for AliceCoins and does not have any left to sell, she creates new ones to sell. If Carol wants to redeem 5 AliceCoins, Alice withdraws \$5 USD and exchanges it with Carol, taking those AliceCoins out of circulation. Alice frequently publishes bank statements showing that her account holds enough USD to redeem all coins in circulation (the number of AliceCoins can be checked anytime on Ethereum).} \\ \hline

\multicolumn{1}{|p{\textwidth}|}{\textbf{Directly Backed.} \newline \footnotesize  Again, Alice is a trusted third party that issues 1000 AliceCoins as ERC20 tokens. She asks \$1 USD for 1 AliceCoin and promises to deposit and hold the payment in a bank account. As before, Alice creates new AliceCoins when she runs out and publishes frequent bank statements. Unlike above, she offers no direct redemption of AliceCoins for USD.} \\ \hline

\multicolumn{1}{|p{\textwidth}|}{\textbf{Indirectly Backed.} \newline \footnotesize Alice is no longer assumed to be trustworthy. She sets up a DApp that can hold ETH and issue tokens. The DApp determines how much ETH is equivalent to \$1.50 USD using the current exchange rate, provided to the DApp by a trusted third party oracle, and Alice deposits this amount of ETH into the DApp. The DApp issues to Alice two places in a line --- each place is a transferrable token. At some future time, the holder of the first place in line can redeem up to \$1.00 USD worth of the deposited ETH at the future exchange rate, and the holder of the second place in line gets any remaining ETH. Alice will transfer the first place in line (as a stable coin called AliceCoin) to Bob for \$1.00 USD, and will hold or sell the second place in line. When Bob redeems the AliceCoin, it will be worth \$1 USD in ETH when the entire deposit of ETH is worth more than \$1 USD. If the exchange rate drops enough, the deposit will be worth less than \$1 USD --- Bob will get all of the deposit and the holder of the second place in line will get nothing.} \\ \hline

\multicolumn{1}{|p{\textwidth}|}{\textbf{Money Supply Adjustments.} \newline \footnotesize Alice forks Bitcoin to create a new altcoin called AliceCoin. She tweaks the schedule for releasing new AliceCoins (called the coinbase amount in Bitcoin) according to the rules outlined below. She sets up a trusted oracle for the latest exchange rate of AliceCoins to USD. AliceCoin is programmed to apply an intervention when the price of an AliceCoin exceeds \$1.02 USD or dips below \$0.98 USD. If the price exceeds \$1.02 USD, the miner is allowed to increase the coinbase amount (the amount is determined by some mathematical relationship with how much the price exceeds \$1.02 USD). If the price dips under \$0.98 USD, the miner must decrease the coinbase amount based on the same relationship. The correctness of the claimed coinbase is verified by other miners in deciding to accept or reject a mined block, as per all other checked conditions in Bitcoin.} \\ \hline

\multicolumn{1}{|p{\textwidth}|}{\textbf{Asset Transfer.} \newline \footnotesize Alice instantiates a DApp with an ERC20 token called AliceCoin. The DApp is programmed to apply an intervention when the price of an AliceCoin exceeds \$1.02 USD or dips below \$0.98 USD according to a trusted oracle. If the price exceeds \$1.02 USD, the DApp creates new a set of AliceCoins (as above, according to some mathematical relationship) and transfers them to users waiting in line for them. How do users wait in line? When the price dips under \$0.98 USD, the DApp creates new positions at the end of the line and auctions them off to the highest bidder. The payment for a place in line is made in AliceCoins from the bidder to the DApp and the DApp destroys the payment. The place in line is a transferrable token. If the line is empty, AliceCoins are distributed according to a fallback policy (see main text).} \\ \hline

\end{tabular}
\caption{Major types of stability mechanisms for stablecoins.\label{tab:mech}}
\end{table*}
