% !TEX root = ../main.tex
\section{Introduction} 
Very quick introduction to blockchain/crypto
What is stability vs volatility.




\section{The current state of the stable coins} % Didem :

%different ways and issues with each.

we can sort stablecoins into three categories: fiat-collateralized, crypto-collateralized, and non-collateralized.

1) Fiat-collateralized stablecoins - These are pegged, generally 1:1, to a fiat currency?they maintain value because every unit of stablecoin is ?backed? by a unit of fiat. So, hypothetically, you can always exchange 1 stablecoin for 1 fiat, and so their values converge. The most popular (though I use this term loosely) fiat-collateralized stablecoin is Tether.
2) Crypto-collateralized stablecoins - These use smart contracts to ?lock up? other cryptocurrencies in a smart contract, which then backs the stablecoin. While the stablecoin in this case has to be overcollateralized to compensate for other cryptocurrencies? volatility, in theory you can always liquidate your stablecoin for other cryptocurrency. MakerDAO is the most prominent of these stablecoins.
3) Non-collateralized stablecoins - These algorithmically manipulate supply and demand to maintain the price. In essence, it acts as an automated central bank, a rudder that manns itself. Basis is the early leader in this pack.





\section{Problems that stable coin addresses} %Mahsa:
% + their alternative solutions:
\subsection{Government Surveillance}
Stablecoins carry several advantages over govtcoins. For one, since stablecoins are not ?issued? by nation states, users do not have to worry about government surveillance. In countries helmed by oppressive governments, decentralized stablecoins could still be popular for this reason. Additionally, it?s reasonable to project that if governments can program their currencies, taxes will be built into the govtcoin. If people or entities wish to avoid taxes, they could choose to stash their capital in decentralized stablecoins over currency havens or other govtcoins (though this is obviously not recommended - ?always pay your doctor and the IRS?). Finally, stablecoins also offer potentially easier access as governments might attempt to restrict ownership of their cryptocurrencies to own-country citizens.

\subsection{Lending}
~\textblue{Talk about lending and other possible problems that stable coins may solve. }

One of the main challenges observed in lending is  monetary instability~\cite{okoyetoward}. The volatile nature of fiat currencies or the cryptocurrencies pose a risk to both the cash taker and the cash provider. The increase in the value of the currency will cause the cash taker to owe more than the initial amount. On the other hand, if there is a decrease in the value, the cash provider will end up receiving an amount less than he/she lent.



\textit{Other possible solutions to lending:} ~\textblue{While stable coin is one of them what can be the other possible solutions?}

\textblue{Taken from ~\cite{okoyetoward}, paraphrasing needed!}
Addressing monetary instability:
\begin{itemize}
	\item The rate of release of new currency into the system could be modified to en-
	able new currency to be introduced at (i) a more insightful rate or (ii) based
	on some internal metrics of the system like number of transactions. [Remark:
	an insightful rate has been elusive despite many alt-coins customizing the
	schedule and it is difficult to see how metrics could not be gamed].
	\item A cryptocurrency can also use explicit pegging but it is no better suited to this system than standard currencies. 
	\item A central bank could manage currency circulation while allowing other as-
	pects to be decentralized. [	Remark:Central banks have been historically
	unsuccessful at using money circulation as a target].
	\item The loan could be use the cryptocurrency as the medium of exchange but
	use a stable (e.g.,	government) currency as the unit of account.
\end{itemize}

\textit{Other possible solutions to lending:} While stable coin is one of them what can be the other possible solutions?

\section{Comparison Framework} % Put the code for the framework
Define the properties that are considered during the design of stable coins.
~\textblue{Collateralization info from ~\cite{bitmex}, decide which projects to choose that exemplify each category best.}
\begin{table}[]
	\begin{tabular}{|l|l|l|l|}
		\hline
		& Collateralization (+ the value of the collateral) & Price Oracle   \\ \hline
		BitShares (BitUSD) &  Crypto-collateralized &  -  \\ \hline
		BitBay & Non-collateralized &   \\ \hline
		 &   &   \\ \hline
	\end{tabular}
\end{table}

\section{Discussion}
%%https://gemini.com/wp-content/themes/gemini/assets/img/dollar/gemini-dollar-whitepaper.pdf -> ref


