% !TEX root = ../main.tex
\section{Introduction}
Very quick introduction to blockchain/crypto
What is stability vs volatility.
About coin supply and volatility~\cite{sams2015note}.
Short-term stability is important for transactions and long-term stability is important for holding`\cite{forbes}.


\section{The current state of the stable coins} % Didem :

%different ways and issues with each.

we can sort stablecoins into three categories: fiat-collateralized, crypto-collateralized, and non-collateralized.

1) Fiat-collateralized stablecoins - These are pegged, generally 1:1, to a fiat currency they maintain value because every unit of stablecoin is backed by a unit of fiat. So, hypothetically, you can always exchange 1 stablecoin for 1 fiat, and so their values converge. The most popular (though I use this term loosely) fiat-collateralized stablecoin is Tether.

stablecoin projects pegged to national fiat currencies other than the U.S. dollar would also be of value for IBM’s blockchain integration~\cite{cointelegraph}.

Ex: dollar-pegged tokens(Tether and TrueUSD)
This type of stable coins need to rely on a third party, as the tokens need to be backed up by the money that is kept in a certain account~\cite{techrev}. Centralised cryptocurrencies backed by gold (slow increas in the value). ex: Digix Gold Token

Discussion about centralization: "Centralised monetary systems can refuse to give you your money. Tether: "redemptions will not be unreasonably denied, but we reserve the right to selectively deny redemption and creation of Tethers on a case-by-case basis." Globcoin: less reliance on US dollars but still centralized.~\cite{cryptoinsider}

2) Crypto-collateralized stablecoins - These use smart contracts to ?lock up? other cryptocurrencies in a smart contract, which then backs the stablecoin. While the stablecoin in this case has to be overcollateralized to compensate for other cryptocurrencies? volatility, in theory you can always liquidate your stablecoin for other cryptocurrency. MakerDAO is the most prominent of these stablecoins.

‘Reserve’ that maintains a peg by using a smart contract to lock up other crypto assets, rather than fiat~\cite{cointelegraph}.

stablecoins using this method are over-collateralized~\cite{overview}.

3) Non-collateralized stablecoins - These algorithmically manipulate supply and demand to maintain the price. In essence, it acts as an automated central bank, a rudder that manns itself. Basis is the early leader in this pack.

The dual-token model — stablecoins and shares One token is stable, and the other is used to whip it back onto the narrow track. Roughly, the idea is that supply of the stable token is dynamically increased and decreased. ex: Basis, Carbon. Then the dual-token model has two flavours: the kind where the stable token is created by locking up other crypto (e.g. DAI and Havven), and the kind where it is not (e.g. Basis, Carbon). ~\cite{cryptoinsider}

The main non-collateralized approach is the seigniorage shares method. The seigniorage shares method uses smart contracts that automatically expand and contract the supply of the non-collateralized stablecoin using algorithms to maintain its value`\cite{overview}.





\section{Problems that stable coin addresses} %Mahsa:
% + their alternative solutions:
\subsection{Government Surveillance}
Stablecoins carry several advantages over govtcoins. For one, since stablecoins are not ?issued? by nation states, users do not have to worry about government surveillance. In countries helmed by oppressive governments, decentralized stablecoins could still be popular for this reason. Additionally, it?s reasonable to project that if governments can program their currencies, taxes will be built into the govtcoin. If people or entities wish to avoid taxes, they could choose to stash their capital in decentralized stablecoins over currency havens or other govtcoins (though this is obviously not recommended - ?always pay your doctor and the IRS?). Finally, stablecoins also offer potentially easier access as governments might attempt to restrict ownership of their cryptocurrencies to own-country citizens.

\subsection{Lending}
~\textblue{Talk about lending and other possible problems that stable coins may solve. }

One of the main challenges observed in lending is  monetary instability~\cite{okoyetoward}. The volatile nature of fiat currencies or the cryptocurrencies pose a risk to both the cash taker and the cash provider. The increase in the value of the currency will cause the cash taker to owe more than the initial amount. On the other hand, if there is a decrease in the value, the cash provider will end up receiving an amount less than he/she lent.



\textit{Other possible solutions to lending:} ~\textblue{While stable coin is one of them what can be the other possible solutions?}

\textblue{Taken from ~\cite{okoyetoward}, paraphrasing needed!}
Addressing monetary instability:
\begin{itemize}
	\item The rate of release of new currency into the system could be modified to en-
	able new currency to be introduced at (i) a more insightful rate or (ii) based
	on some internal metrics of the system like number of transactions. [Remark:
	an insightful rate has been elusive despite many alt-coins customizing the
	schedule and it is difficult to see how metrics could not be gamed].
	\item A cryptocurrency can also use explicit pegging but it is no better suited to this system than standard currencies.
	\item A central bank could manage currency circulation while allowing other as-
	pects to be decentralized. [	Remark:Central banks have been historically
	unsuccessful at using money circulation as a target].
	\item The loan could be use the cryptocurrency as the medium of exchange but
	use a stable (e.g.,	government) currency as the unit of account.
\end{itemize}

\textit{Other possible solutions to lending:} While stable coin is one of them what can be the other possible solutions?

\section{Comparison Framework} % Put the code for the framework
Define the properties that are considered during the design of stable coins.
~\textblue{Collateralization info from ~\cite{bitmex}, decide which projects to choose that exemplify each category best.}
\begin{table}[]
	\begin{tabular}{|l|l|l|l|l|}
		\hline
		& Collateralization (+ the value of the collateral) & Price Oracle & Centralization  \\ \hline
		BitShares (BitUSD) &  Crypto-collateralized &  No & \\ \hline
		BitBay & Non-collateralized & &  \\ \hline
		 DAI& Crypto-collateralized (ETH)  &  Yes& \\ \hline
		 BitShares&Crypto-collateralized&Yes&\\ \hline
		 Basis&Non-collateralized&& \\ \hline
		 Tether & Fiat-collateralized (USD) & & Yes \\ \hline
		  &&& \\ \hline
	\end{tabular}
\end{table}

Decentral price oracle and Schelling point~\cite{cryptoinsider}
DAI and Bitshares~\cite{cryptoinsider}

\section{Discussion}
%%https://gemini.com/wp-content/themes/gemini/assets/img/dollar/gemini-dollar-whitepaper.pdf -> ref


