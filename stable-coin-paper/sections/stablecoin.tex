% !TEX root = ../main.tex
%==========================================%

\section{Introduction}\label{sec:Intro}

Many early cryptocurrency proposals designed secure digital representations of government-issued money (which cryptocurrency enthusiasts typically call `fiat'). While Bitcoin was not the first proposal for a digital currency that is issued and operates independently of existing currencies and financial infrastructure, Bitcoin~\cite{nakamoto2008bitcoin} is the first of this type to establish wide-scale deployment. Without government oversight, the exchange rate of Bitcoin is essentially subject to: (a) an algorithm which releases new BTC (Bitcoin's currency) on a fixed schedule, and (b) the market for exchanging Bitcoin for other things of value, namely fiat currencies such as the USD, and potentially (c) the market for participating in transaction validation which is integral into how new BTC comes into circulation.

%========================%
\begin{figure}[t]
	\centering
	\includegraphics[width=0.8\textwidth]{figures/allCurrencies.pdf}
	\caption{\label{fig:btcandfiat}Comparison among fiat currencies and Bitcoin: The values are retrieved daily between  01 Jan 2016 and 01 Jan 2019. Note that 1000 mBTC = 1 BTC.}
\end{figure}
%========================%

From the inception of exchanges for buying and selling BTC for USD in 2010 to the time of writing, the exchange rate of BTC with the USD has been marked by extremely volatile with large fluctuations in its value that are atypical of a government-managed currency. Figure~\ref{fig:btcandfiat} illustrates this volatility by plotting the exchange rate of BTC (with the USD) alongside the same exchange rate for three economic zones---Europe, UK, and Canada---which all appear relatively stable. Note that Figure~\ref{fig:btcandfiat} deliberately includes the UK's referendum on exiting the EU (`Brexit') in June 2016, which was followed by a `sharp decline' and `volatility' in GBP's exchange rate.\footnote{Descriptions from the following \textit{BBC} articles: ``The markets facing trading turmoil'' (27 Jun 2016) and ``How does Brexit affect the pound?'' (15 Jan 2019).}  Relative to BTC however, this `severe swing' looks like a mild pinch of GBP's exchange rate with EUR in Figure~\ref{fig:btcandfiat}.

In response to Bitcoin's extreme volatility, a flood of proposals have been made for alternative designs that would offer a more stable exchange rate (called `stablecoins') between the newly proposed stablecoin and a government-issued currency like the USD. Broadly, the proposals can be split into two categories: ones that essentially create a digital representation of a currency that can be transacted like a cryptocurrency, and ones that propose separate currencies with some mechanism for stability and/or intervention built into the design.

\paragraph{Contributions.} This paper is essentially a survey of work on stablecoins but we aim at making a number of subtle research contributions to ensure this survey is actually useful to the reader. First and foremost, we are very selective in the concepts from finance we bring into the survey and explain each from first principles, while attempting to minimize or eliminate jargon. Next we distill stablecoin proposals down to a set of fundamental primitives and describe these concepts rather than enumerating the intricate details of how particular `brands' of stablecoins work---details that could change tomorrow. That said, we do provide, as the reader probably expects, a chart mapping existing stablecoin brands into our categorization. Additionally, we also consider the question and potential for the stability of  index-cryptocurrencies (namely gas which is used in Ethereum), which are very pertinent to a discussion of stablecoins, yet not typically addressed. Last, we offer a novel visualization style for exchange rates we have not seen before used for exchange rates.


% Most of the coins are mostly focusing on the supporting infrastructure, maybe we also have to talk about the supporting infrastructure of the stablecoin
% Talk about the crypto custodianship somewhere in the paper


\section{Related Work}

Other blog posts, whitepapers, etc on systemizing stablecoins.

%==========================================%

%Mahsa:

%In 2018, Bitmax published a report in which they briefly discuss the two types of stablecoin (\ie BitShares (BitUSD) and MakerDAO (Dai)) and provided a comparison in terms of different design decisions they apply~\cite{Bitmex}.
%The idea seigniorage shares was first introduced by Robert Sams in 2017, where he proposed a new dual token system that uses two coins to achieve price stability: (i) a stablecoin (\eg fiat currency) and (ii) a volatile coin (\eg equity shares)~\cite{sams2015note}.
%In "The Denationalization of Money, F. A. Hayek, the Nobel Prize-winning economist, discussed about government authority over money supply. He argued that money should not be supplied by governments and instead, it should be market driven and based on market's supply and demand~\cite{F. A. Hayek}.
%In 2016, Ametrano introduced "Hayek Money", a new monetary policy of elastic non-discretionary supply that can be used to achieve a price stable cryptocurrency~\cite{ametrano2016hayek}.
%In one of his blog post, Buterin has discussed .... (https://blog.ethereum.org/2014/11/11/search-stable-cryptocurrency/) ~\cite{TheSearc7:online}
%
%Should we talk about the followings:
% https://static1.squarespace.com/static/55f73743e4b051cfcc0b02cf/t/58c7f80c2e69cf24220d335e/1489500174018/R3+Report-+Fedcoin.pdf ?
% http://hardjono.mit.edu/sites/default/files/documents/Digital-Trade-Coin.pdf
%
%@article{sams2015note,
%  title={A Note on Cryptocurrency Stabilisation: Seigniorage Shares},
%  author={Sams, Robert},
%  journal={Brave New Coin},
%  pages={1--8},
%  year={2015}
%}
%
%@book{F. A. Hayek,
%  title={The Denationalization of Money},
%  author={Friedrich Hayek},
%  pages={},
%  year={1978},
%  publisher={Institute of Economic Affairs}
%}
%
%@article{ametrano2016hayek,
%  title={Hayek money: The cryptocurrency price stability solution},
%  author={Ametrano, Ferdinando M},
%  year={2016}
%}
%@misc{TheSearc7:online,
%author = {Vitalik Buterin},
%title = {The Search for a Stable Cryptocurrency},
%howpublished = {\url{https://blog.ethereum.org/2014/11/11/search-stable-cryptocurrency/}},
%month = {November4},
%year = {2014},
%note = {(Accessed on 02/11/2019)}
%}
%=====================Gold vs ETFGold  ===================%
%Gold ETF: Investors want  exposure to the asset that’s electronically traded without taking possession of the commodity.
%advs
% Easy to trade
% No storage and insurance cost
% Hard to steal
% Gold ETF is the most convenient way to store money in gold.
% It is both convenient and low cost. If you’re looking for an inexpensive way to invest in the direction of the gold price, GLD is ideal.
% disadv:
% Ownership of Gold ETFs does not entitle investors to any physical gold. 
%  in special occasions when the  market or the bank is not available, the user does not have access to to trade his shares. In this case, ownership of ETF is not any useful for the user.
%  Gold EtTF has counter-party risk and exposed to bad management as the investors have to trust a bank or other trustee to hold and manage their shares
% Physical Gold: Investors want to own the tangible asset
%advs:
% It’s a solid investment and serves as a great hedge against inflation. With strong appreciation through the years, it has proven reliably positive returns even during economic downtimes. Gold prices fluctuate but won’t go away, it won’t be devalued, and it will always be valuable anywhere in the world.
% When you invest in physical gold, you’re entirely in charge of your investment, meaning you’re not trusting it to someone else like an ETF. You decide how to store and secure it.
% When you sell a gold ETF, that data is required by law to be immediately transmitted to the government by the brokerage. Sell gold bullion or bars, and you must report capital gains on your income tax return.
%disadv:
% When you decide to sell your gold, you will have to pay a dealer’s fee as a percentage of your purchase. Hard to buy and sell (trade)
% easy to be stolen
% hard to store


%==========================================%
%Didem:
%Bitmex provides a classification for the stablecoins based on the fact whether there is an underlying collateral or not~\cite{bitmex}. They focus on the distributed stablecoins. We take centralized projects into consideration too.

%The report from Blockchain provides an extensive classification of 57 stablecoins ~\cite{report}. Besides grouping them according to the mechanism that provides stability, they also provide information about the issues related to governance like legal structure, investors, partners etc.

%Consensys also provides a classification base on the same framework.(https://media.consensys.net/the-state-of-stablecoins-2018-79ccb9988e63). They have a discussion decentralization too.

%In~\cite{cryptoinsider}, the chosen projects are analyzed under two broad categories: centralized an decentralized. Centralized ones are further grouped into different categories according to the type of asset that is used to back the stablecoins.

%~\cite{hackernoon},~\cite{comprehensiveOverview},~\cite{linkedin},~\cite{coinsexplained} and ~\cite{overview} categorize the stablecoins under three categories: fiat-collateralized, crypto-collateralized and non-collateralized.

%(http://www.andrew.cmu.edu/user/azj/research/rzj_slides201808.pdf): three categories: 100\% USD Reserve backed coins, Protocol coins without redemption, Protocol coins with redemption in floating-rate crypto-currencies

%(https://icosbull.com/whitepapers/2265/USDX_Protocol_whitepaper.pdf) three generations of stablecoins: Collateral-backed IOU, Collateral-backed on blockchain, Elastic monetary supply based. (This is a whitepaper for a protocol but it has a different classification)

%Compared to aforementioned works that provide a systemization of the stablecoins, we provide a different framework to classify the existing mechanisms to achieve stability. While most of the works divide stablecoins into three categories (fiat-collateralized, crypto-collateralized and non-collaterlized), we divide them into two broad categories as backed and intervention. Then, we further group them under different mechanisms.

%(https://multicoin.capital/2018/01/17/an-overview-of-stablecoins/)  Stablecoins are discussed under three main categories as centralized IOU issuance, collateral backed, and seigniorage shares along with the discussion of different types of oracles and the challenges that stablecoins face.

%How we chose the projects: We performed a search query on \texttt{coindesk.com} and classified the projects that are mentioned in the posts until January 11, 2019. \footnote{https://www.coindesk.com/} (Copied the same text here!)


%==========================================%

\section{Preliminaries}

\subsection{Prices}

If 1 BTC is worth \$3598.76 USD, as Google says it is at the time of writing, what does that actually mean? There are several subtleties here: (1) what that price actually represents, (2) the relationship between a quoted price and its actual price, (3) the concept that prices are really an exchange of one type of valuable good for another, and (4) the distinction between something's price and its value. The quoted price means that two (hopefully different\footnote{A trade between the same person is called a wash trade and is illegal in most regulated markets.}) people recently exchanged BTC and USD at a valuation of 1 BTC for \$3598.76 USD. First, note that it does not necessarily mean that exactly 1 BTC was exchanged --- it could have been 1 mBTC for \$3.60 or 1000 BTC for \$36M USD. Further, this valuation on the previous trade does not mean you will necessarily be able to exchange 1 BTC for \$3598.76 USD. Last sale price is an indicator of current price that becomes stale as time between subsequent exchanges increase (for example, for a house that last sold 30 years ago, last sale price on a house is not a good indicator of current price).

Instead, we will use the idea of that a cryptocurrency (or any asset) has two prices: (1) the most someone is willing to pay and (2) the least someone is willing to sell for. These are referred to as the best bid price and best ask (or offer) price respectively. Note that the best bid price should logically be less than the best ask price, otherwise an exchange would happen (such prices might occasionally `cross' but this should be temporal and quickly resolved with an exchange). The spread between these prices is called the bid-ask spread. 

To understand why this is relevant to stablecoins, consider an example. Say a stablecoin is designed to ensure one unit is always priced at \$1 USD. To argue stability, one must show both that (1) the bid price should never exceed \$1 dollar and (2) the offer price should never dip below \$1 USD. Note, conversely, that bids can dip below \$1 USD (everyone prefers to pay less than something is worth) and asks can exceed \$1 USD (everyone prefers to receive more than something is worth).

\subsection{Exchange Rates} 

Consider that several hours after writing the previous section, 1 BTC is now priced at \$3566.56 USD. In one sense, the price of BTC decreased by \$32.20. However it is exactly equivalent to say the price of \$1 USD increased by 0.002 mBTC. This raises a natural question: did BTC decrease in price or did USD increase in price? With an exchange rate, it is impossible to tell. We only know that the price of BTC and USD became closer in price over this short period of time.

To determine which currency is moving, one might consider a third or forth currencies to try and triangulate if BTC is moving in price, or USD is moving in price, or both. For example, in Figure~\ref{fig:btcandfiat} it certainly appears that BTC is the currency that is moving because the rest of the currencies are stable relative to each other. The only alternative is that USD, EUR, GBP, and CAD are volatile currencies that move together as a cluster relative to the stability of BTC. But it is much simpler to conclude that BTC is moving. 

%========================%
\begin{figure}[t]
	\centering
	\subfloat[GBP with respect to EUR and USD]{\includegraphics[width=0.75\textwidth]{figures/gbpBrexit.pdf}\label{fig:brexit}}
	\hfill
	\subfloat[Legend]{\includegraphics[width=0.25\textwidth]{figures/compass}\label{fig:legend}}
	\hfill
	\subfloat[The simpliest interpretation of the plots where $X$ refers to the currency on the x-axis and likewise $Y$.]{
	\begin{tabular}{|c|l|}
\hline
\textbf{Direction} & \textbf{Interpretation}   \\ \hline

         1/5                & $Y$ is losing (1) / gaining (5) value \\ \hline
         2/6                & Plotted asset is gaining (2) / losing (6) value \\ \hline
         3/7                & $X$ is losing (3) / gaining (7) value \\ \hline
         4/8                & \multicolumn{1}{p{14cm}|}{Plotted asset is gaining (4) / losing (8) value against $X$, while losing (4) / gaining (8) value against $Y$} \\ \hline

\end{tabular}
	}
	\caption{Brexit's effect on GBP.\label{fig:Comparison}}
\end{figure}
%========================%

In order to apply this same logic in a visual way, we have created a number of charts like the one provided in Figure~\ref{fig:brexit}. Unlike most exchange rate graphs, these do not use a time axis. Instead each axis is a reference currency. In this case, the price of GBP (plotted value) in USD (x-axis) and EUR (y-axis) forms a coordinate. For the last day of each month, a new coordinate is added and joined with a line from the previous value. This is inspired by similar charts on the website FiveThiryEight for things like kicking distance in football\footnote{``The 52 Best---And Weirdest---Charts We Made In 2016,'' \textit{FiveThirtyEight}, 30 Dec 2016.} and they appear to be called connected scatter plots.

Lines in a connected scatter plot can move in any direction. Figure~\ref{fig:legend} shows how we number the directions from 1 (upward or due north) clockwise to 8 (north-west). For each direction, we describe the simplest interpretation of what that price direction means. By simplest, we mean specifically that we keep an explanation that involves a single currency moving rather than an explanation that involves a pair of currencies moving in tandem. For example, in Figure~\ref{fig:brexit}, GBP shows a drastic movement along direction 6 starting at the time period marked Brexit. This means that GBP is losing value against both EUR and USD. The simplest explanation is that the movements are originating from GBP which is consistent with it losing value after Brexit. Later, GBP shows a lot of horizontal movements along the 7/3 line. The simplest explanation for this segment is volatility in USD rather than GBP. 

We will return to these charts later in Section~\ref{sec:stability} where we will use a government currency as one reference (USD on the x-axis) and a cryptocurrency as the other reference (BTC on the y-axis). A stablecoin should exhibit mostly vertical movements along the 1/5 direction. 

\subsection{Valuation}

Recall that in the previous section, 1 BTC was priced at \$3566.56 USD. This simply means that two people recently swapped some amount of BTC and USD for the stated valuation. Does this mean 1 BTC is worth \$3566.56 USD? Value can mean different things in different contexts. The market value of a currency does present one type of value --- its replacement value, or the cost in USD to replace it. Note that more technically, one should  determine replacement value from the set of best offer prices sufficient to cover the volume of BTC being valued.

But does this mean that BTC is fundamentally worth \$3566.56 USD. This is unlikely because by the time you read this paper, the price of BTC in USD is probably quite different from this quoted value (perhaps humorously so). So what constitutes fundamental value? And why do prices change over time?

Stocks, which represent ownership in a firm, and thus a stake in the firm's equity. Therefore shares (called equities) have a fundamental value called its book value: simplified, it is the firm's capital or equity (the value of its assets minus the value of its liabilities, as reported on its annual audited financial statement) divided by the number of outstanding shares. Working in reverse, the price of a single share multiplied by the number of shares represents the market capitalization of the firm. In theory, these numbers should be the same but often are not. When the market capitalization exceeds the reported capital, the market believes the firm's capital will increase over time. If the market capitalization is less than the reported capital, it demonstrates a lack of confidence in the soundness of the firm's financial statements. Floating currencies like the USD, EUR, GBP, and CAD do not have the equivalent of a book value.

To explain Bitcoin's exchange rate with fiat currencies, an oft-repeated theory has emerged that attributes Bitcoin's value to the hydro consumed by blockchain mining. While imprecise, the theory suggests that if a valuable resource $x$ is consumed to produce $y$, the value of $x$ is imparted into $y$. Setting aside the nuance that the hydro contributed to the Bitcoin system only indirectly produces new coins (it produces blocks, and blocks produce coins only for now), there is no economic principle underlying this transfer of value.

%If Alice goes to Peter Luger's in Brooklyn, consumes a \$100 ribeye, and mints a literal shitcoin out of the result --- is that coin worth \$100 because it is ``backed'' by \$100 worth of steak?


\subsection{Stability and Volatility}

When the price of a currency changes over time, it is due to one of two reasons: new information about the currency's fundamental value (even if we cannot concretely say what it is) or transitory volatility due to the trading activites of uninformed traders~\cite{harris2003trading}. For a government issued currencies, information like national inflation rates, macro-economic policies, changes in trade flows, and changes in capital flows seem predictive of changes in the value of the country's currency~\cite{harris2003trading}. Note that a cryptocurrency, like Bitcoin, has none of these indicators. While volatility can be measured mathematically (using variances or deviations), most stablecoins do not offer a concrete, positive definition of what stability means. They tend to be defined by a negative sentiment of what they do not want (the volatility of Bitcoin) rather than a positive sentiment of what they do want.

\subsection{Functions of Money}
 
There is controversy over whether Bitcoin, and other cryptocurrencies, can even be classified as currencies. The original intention was for Bitcoin to be a currency, however it has been classified in many different ways: from a digitally scarce commodity, to a speculative instrument, to an entirely new asset class. Most introductory finance textbooks classify currencies according to a set of properties it should fulfill for its users. It should operate as a medium of exchange, which roughly means that Alice will accept the currency from Bob because she is confident Carol will later accept it from her. Given the existence of exchange services, Bitcoin is generally considered an acceptable medium of exchange (albeit with some friction). Next, a currency is useful when it servers as the unit of account for pricing other assets. Bitcoin is almost never used as a unit of account and if goods are sold for Bitcoin, it is often priced in, say, USD with a short-lived spot conversion of the price to BTC for Bitcoin purchases. Finally, currencies should represent a stable store of value. Alice will not accept a currency that depreciates quickly in value from Bob because even though Carol might accept it, what she can obtain from Carol in exchange will be worth less. Less intuitively, currencies that appreciate quickly in value are equally problematic. Alice might gladly accept it from Bob but Bob is unlikely to part with it, and so currencies like this tend to be hoarded. They also hamper lending (see next section).

The goal of a stablecoin is to add the store of value feature to cryptocurrencies, which are already a somewhat adequate medium of exchange. Further, if the currency is stable, it may become a more prominent to use it as a unit of account. Thus stablecoins are intended to make cryptocurrencies more currency-like.

 %Cite Curse of Cash

\subsection{Lending}

Despite quite a few blockchain applications in financial technologies, there has been little deployment of lending.
Lending is difficult to be deployed on the blockchains, bdue to the monetary instability observed in the existing cryptocurrencies~\cite{okoyetoward}. This volatility has led the cryptocurrencies to be used more as speculative investments instead of serving as store of value and unit of account. In a lending situation with volatile currencies, where their values are being depreciated or appreciated over time, the cash taker will eventually owe more than what he has borrowed or the vice versa. Therefore, the volatility in the value of cryptocurrencies causes serious concerns and difficulties both for cash takers and cash providers~\cite{okoyetoward}. In contrast, lending perfectly works if a loan is done with a stable cryptocurrency, whose value remains stable over the time.


%==========================================%

\section{Systemization of stablecoins and Justification}

% Two basic concepts: (1) a tokenization of a fiat currency (say a digital dollar) or any asset or asset portfolio of assets. This is an old idea: liberty reserve, eGold. (2) a low-volatility coin that is tied to any existing asset. For example, with USD as the baseline, Euros are much more stable than Bitcoin. It is not because Euros are backed by USD or have any direct reference to the USD. It is because USD and Euros use the same model, a central bank management, with similar inputs (the interest rates their customers, commercial banks, use when lending cash to each other to meet various legal requirements that help ensure the banks will have enough cash on hand to server their customers and can withstand some of their investments going bad without going bankrupt).


% Dollar tokenization vs. low volatitility

% !TEX root = ../main.tex


%-------------------Fancy Table ----------------------%

% = = = Rotated Table Entry \headrow

%\usepackage{adjustbox}
%\newcommand{\headrow}[1]{\multicolumn{1}{c}{\adjustbox{angle=45,lap=\width-0.5em}{#1}}}

% = = = Table bullets: \full and \prt (full and part)

\newcommand{\full}{$\bullet$}
\newcommand{\prt}{$\circ$}
% ------------------------------------------------------------------------------------------------------------------------------------------------%
%Stablecoin projects%

\definecolor{UnitedNationBlue}{rgb}{0.30,0.53,1}
\definecolor{LightSteelBlue}{rgb}{0.69,0.77,0.87}
\definecolor{LightGrey}{rgb}{0.83,0.83,0.83}


\begin{table}[h!]
\centering

\begin{tabular}{|l|l|l|l|l|}

\hline
\rowcolor{lightgray}
\textbf{Category} & \textbf{Sub-Category} & \textbf{Examples} & Rank \\  \hline

\multirow{16}{*}{Backed}		& \multirow{8}{*}{Directly-Backed \& Redeemable}					& \textbf{USDC} & 20 \\ \cline{3-4}
						&														& TrueUSD & 26 \\ \cline{3-4}	
						&														& Paxos & 38 \\ \cline{3-4}												
						&														& Gemini Dollar & 52 \\ \cline{3-4}
						&														& StableUSD (USDS) & 685 \\ \cline{3-4}
						&														& Stronghold USD & 891 \\ \cline{3-4}
						&														& Petro & 1210 \\ \cline{3-4}
						&														& \multicolumn{1}{p{5cm}|}{Ekon, WBTC, emparta} & $\perp$ \\ \cline{2-4}
						& \multirow{6}{*}{Directly-Backed}  								& \textbf{Tether} & 6 \\ \cline{3-4}
						&														& EURSToken & 95 \\ \cline{3-4}
						&														& BitCNY & 304 \\ \cline{3-4}
						&														& Terracoin & 1280 \\ \cline{3-4}
						&														& Saga & 1495 \\  \cline{3-4}
						&														& \multicolumn{1}{p{5cm}|}{GJY, Novatti AUD, UPUSD} & $\perp$ \\ \cline{2-4} 						
						%& \multirow{1}{*}{Non-Redeemable and Indirectly-Backed}  			&  &  \\ \cline{2-4}
						& \multirow{2}{*}{Indirectly-Backed}								& \textbf{Dai} & 57 \\ \cline{3-4}
                                                &														& BitUSD & 398 \\  \cline{1-4}
 
                                                                                                                 
                                                                                                                 
                                                                                                                 
\multirow{5}{*}{Intervention}                                                           
						& \multirow{1}{*}{Market} 										& Nubits & 892 \\ \cline{2-4}
						& \multirow{1}{*}{Algorithmic with subjective external information}  		& Nomin* & $\perp$  \\ \cline{2-4}
						& \multirow{2}{*}{Algorithmic with objective external information}  		& CarbonUSD* & 1262 \\ \cline{3-4}
						&														& Ampleforth* & $\perp$ \\ \cline{2-4}
						& \multirow{1}{*}{Algorithmic with only internal information} 			& Basis & $\perp$ \\ \hline

\end{tabular}
\vspace{1em}
\caption{\footnotesize{Systemization the current stablecoin projects} on January 11, 2019. The Projects that are in bold are discussed in this paper.}
\label{tab:stablecoins}
\vspace{3em}
\end{table}
% ------------------------------------------------------------------------------------------------------------------------------------------------%

% Jeremy: How do we make a stable coin.
% Jeremy: Table here

\section{Backed}
% issue: the stablecoin is valid as long as the holder trusts the company/person who keeps the reserve, once the trust is gone, the stablecoin worths nothing
%adv:
 	%easy to implement
	% value should matched exactly (eg to USD)
%disadv:
	% trust 3rd party to hold it
	% need additional 3rd party to audit
	% expensive and slow to audit


\subsection{Redeemable and Directly-Backed.}
% Description of how it works: Deposit in bank, etc.
% Regular audits are needed to ensure that the stablecoin is indeed fully collateralized. Although it's redeemable, a few people can redeem it and that itself shows that they have enough reserve, BUT what if nobody wants to redeem it?
% Justification that bids will never exceed $1: an arbitrageur will pay $1 to mint 1 XSC and the 1 XSC for the bid
% Justification that offers will be less than $1: an arbitrageur will purchase the 1 XSC and redeem it for $1
% Risk: not redeemable -> redemption is not 100%, the coin will be offered at less than $1. E.g., offers of $0.50, says 50% it can't be redeemed.
% Examples: Gemini
% Mahsa: Should we talk about these two recent incidents in this section?
% According to the recent blogpost in  (https://www.coindesk.com/winklevoss-crypto-gemini-gusd-stablecoin-redemption) Gemini has closed the account of some users and in some cases do not let redemption (some KYC issues) +     Paxos has has the same issue, although they let the user redeem their assets and then closed their account (https://www.ccn.com/paxos-standard-hassling-ethereum-traders-trying-to-redeem-stablecoin-pax-for-dollars)


\subsection{Non-Redeemable and Directly-Backed.}
% Description of how it works. Deposit in bank, etc.
% Regular audits are needed to ensure that the stablecoin is indeed fully collateralized
% Justification that bids will never exceed $1:
% Justification that offers will be less than $1
% Examples: Tether
% Remarks:
%

\subsection{Non-Redeemable and indirectly-Backed.}
% Description of how it works.
% Justification that bids will never exceed $1
% Justification that offers will be less than $1
% Examples:
% Remarks: Oracles


\subsection{Redeemable and indirectly-Backed.}
% Description of how it works.
% Justification that bids will never exceed $1
% Justification that offers will be less than $1
% Examples:



\section{Intervention-based} % it used to be Algorithmic

% Control system: input (metrics), make a decision, implement the intervention
% We cannot prove that something here works or not. It is all heuristics. We know this because central banks themselves operate on heuristics that change every couple of decades. Even if we cannot say something does work, we can point out a few suggestions that it will not work. Game-able inputs, inputs/interventions that have not worked for currency.

\subsection{Currency Board}
% Description of how it works.
% Justification that bids will never exceed $1
% Justification that offers will be less than $1
% Examples:

%=================%
%IMF's papers on currency boards(https://www.imf.org/external/pubs/ft/pdp/2000/pdp01.pdf):one of the elements in a currency board: fixed exchange rate pegged to a foreign anchor
%aim: monetary stability and low inflation
%currency board needs sufficient backing of base money (=central bank holds sufficient foreign exchange reserve)
%CBA has adjustment mechanism. It reacts automatically to the changes in the foreign exchange outflows and inflows.Interest rates adjust automatically in currency board based systems
%problem when the domestic inflation rate is higher than the inflation rate of the underlying foreign currency: results in overvaluation of the domestic currency

%(https://www.imf.org/external/pubs/ft/wp/wp9808.pdf): Difference between currency board and pegged exchange rate
%=================%
\subsection{Algorithmic with subjective external information}
% Description of how it works.
% Justification that bids will never exceed $1
% Justification that offers will be less than $1
% Examples: RSCoin

\subsection{Algorithmic with objective external information}
% Description of how it works.
% Justification that bids will never exceed $1
% Justification that offers will be less than $1
% Examples:

\subsection{Algorithmic with only internal information}
% Description of how it works.
% Justification that bids will never exceed $1
% Justification that offers will be less than $1
% Examples:

We performed a search query on \texttt{coindesk.com} and found the following projects which are mostly used in every articles published about stablecoins until January 11, 2019. \footnote{https://www.coindesk.com/} Table~\ref{tab:stablecoins} represents these projects and the methods they apply to achieve stability. (we should mention the search term that we used to extract our resources from coindesk)


\textbf{Distribution of articles speaking about stablecoin:} 2014: 2, 2017: 4, 2018: 112, 2019 (up to Jan 11): 4.

%================Notes on unranked projects:==========================%

%CarbonUSD:
%t hybridizing both fiat backed and algorithmic stablecoins, (hybrid fiat-algorithmic approach)
% 1-1 backed with USD
% unlike purely fiat-backed stablecoin, CarbonUSD's novel meta-token structure enables a seamless future transition to an algorithmic stablecoin model once it reaches sufficient scale as a fully fiat-backed token.
% meta-token smart contract is a key innovation that enables carbonusd to eventually transition from full fiat-collateralization without distrupting its liquidity network effects on exchanges and while maintaining the highest standards for regulatory compliance.
% HOW? CarbonUSD is a basket of whitelisted tokens. A token that is whitelisted may be used to create new CarbonUSD, serving as its collateral. Initially, only one token will be whitelisted, a stablecoin that is 1-1 backed with USdollars in a trust account.
% The first whitelisted token, WT0, is a compliant fiat-backed stablecoin where users can deposit and withdrawal real USD. Governing members of the whitelist can decide when CarbonUSD has enough liquidity to safely switch off from full fiat- collateralization.
% they also have regular audits (unlike Tether)
% Redeemable
% Offchain (fiat-backed, crypto-backed) stablecoins are effective as bootstrapping trust and NOT maintaining it scale. HOWEVERM, Onchain (algorithmic) stablecoins are effective at maintaining trust at scale BUT not at bootstrapping it.
%=================%
%NuBits:  https://nubits.com/NuWhitepaper.pdf
% Like Tether, NuBits are pegged to US dollar, with one dollar equivalent to one NuBit.
% NuBits controls its coin's value by getting shareholders to place buy and sell orders at 1$ for NuBits, rewarding them with revenue from sales of NuBits external cryptocurrency, NuShares.
%==========
% NuBits has suffered two big crashes, with extended peg breaking.
% First :
% In 2016, NuBits’ peg infamously broke, and it remained broken for 3 months. The initial price drop happened between May 26th and June 20th, 2016. This was about the same time that Bitcoin’s price suddenly spiked, after 6 months of relative stability. It’s plausible that the drop happened because of the following: People who had capital in NuBits saw how Bitcoin was spiking. They wanted to get in on the spike, so they sold their NuBits in large quantities to buy Bitcoin. The NuBits peg was unable to handle the large sell-offs and broke. The price tanked and the peg stayed broken for an extended period. It seems that when the Bitcoin and volatile cryptoassets’ prices spike, investors with capital in stablecoins will want to sell them off to get in on the spike. That causes strong downward pressure on the stablecoin price.
%==========
% Second:
% After NuBits crashed in 2016, its market cap grew 1,500% on Coinmarketcap between the end of 2017 and the beginning of 2018, going from $950k to $14 million. This is strange, considering the NuBits peg had been broken for such a long period in 2016. Their market cap had been stagnant for years, and suddenly it takes off. Was this spike just an accounting error, or had people suddenly decided NuBits was actually amazing?
% Turns out the increase was real. However, it did not happen in one day. People were buying millions of NuBits in late December. The price of the NuBits stablecoin was notably above its $1 peg between December 20 and December 28, when it peaked close to $1.50(!).
% What caused the increase? It’s clear when one looks at the Bitcoin price history. The NuBits high price period starts when Bitcoin had its “Christmas crash.” By December 22 there was a strong news narrative about Bitcoin crashing.
% As long as there was a lot of uncertainty about Bitcoin’s stability, people kept buying large numbers of NuBits. Then, by the New Year, when Bitcoin temporarily normalized again, the large NuBits buy-ins stopped.
% What happened, then? Well, Bitcoin holders noted the dropping Bitcoin prices and got worried about an imminent crash. Converting BTC to USD is slow and can lead to taxation. So the worried Bitcoin holders converted their crypto assets to stablecoins rather than cash. Some choose NuBits. The very large influx of Bitcoin money doubled the NuBits market cap several times over. The NuBits folks were frantically printing new money and selling it off to the Bitcoin holders. But demand was so great among panicked Bitcoin holders that the new NuBits couldn’t be printed fast enough and their price was driven up high.
% This shows how when Bitcoin and cryptomarkets crash, capital rapidly flows into stablecoins.
%==========
% NuBits sounds very much like Basis: When the price of the Basis token below 1 dollar, they start selling sth called a bond token, and they actually sell this at a price below 1$ with a promise that it can be redeemed for a Basis coin in the future. when the price of Basis token goes above 1$, the Basis algorithm has to increase the money supply, so tehy issue you a 1 to 1 Basis coin for each Basis bond token that you held.
%=================%
%Ampleforth:
% It uses a different method of preserving its dollar-to-unit ration, by transferring volatility from unity price to unit count.
%According to Trail of Bits, a malicious market maker could play with the stability of Ampleforth. the issue is their oracle services use whitelisted sources (just like Dai)
%=================%
%Havven
% is mostly like Dai, they also use over collateralization with 1:5 ratio.
% They use a similar arbitrage as Dai but with different functionalities by locking up Havven tokens or by reverse burning Nomins by reclaiming Havven tokens  to keep the price stable.
% One key difference between Nomin and Dai is the collateral for Dai is Eth but for Nomin is the actual Havven token itself.
%=================%
%Basecoin (Basis Algorithm)
% Non-collateralized
% Controlled by algorithmic supply
% The price  stability of this coin is managed by Basis protocol which is a central bank algorithm that increases or decreases the supply of teh coin based on supply and demand
% Still experimental and not yet operational
 % The coin supple is entirely done based on the information given by differnt oracles that make the protocol aware of condition of supply and demand
 % The peg could be anything: fiat, assets, or a basket of assets

%==========================================%

\section{What Stability looks like}\label{sec:stability}

\section{Investigation of gas volatility}\label{sec:GasInvs}
% Gas: non-redeemable backed by Ether. like putting Ether in the contract and get USD digital token. why non-redeemable? because you burn Gas to get your digital tokens.
% Depending on how much the user wants to pay for ecah unit of Gas, this will determine how expensive the tx is in Ether.
% Gas is internal to the Ethereum blockchain and users cannot hold it, instead they can have Ethere, BUT the conversion rate between Gas and Ether changes over time and that's why there are huge spikes in the average Gas price plots.
% But we can make Gas redeemable, there is a way in Ethereum to do that, the GasToken that uses this strategy. It allows you to stockpile and trade Ethereum Gas.
% Gas is not a currency so you cannot buy things with it BUT you can turn it into a currency : having a smart contract and swaps it with other tokens
% Gas is actually stable with respect to others, the reason it's not obvious is human error and their wrong mental model
% Gas is not pegged to anything
% Gas is free floating but it’s driven (governed) by market forces (auctioning off)
% there is an upper band and lower band for Gas
% So in this section we look into how to trade Gas and its derivatives?

%=======GasToken details=======%
% Operations that modify the global state of the Ethereum blockchain are very expensive \ie writing into a contract storage.
% Ethereum motivates the user to free up space on the blockchain by refunding the Gas if they delete their contracts. This is what GasToken does.
% GasToken is a smart contract that for the first time allows Ethereum users to buy and sell gas by tokenizing gas on the network.
% They allow users to store Gas when it is cheap and use it once its price goes up.
% Gas price refers to the amount of Ether you’re willing to pay for every unit of gas.
% When a user stores cheap Gas, he can later spend that Gas to send a tx, that means now he pays for less gas for that tx to go through.
% GasToken allows a transaction to do the same amount of work and pay for less gas, saving on miner fees and costs and allowing users to bid higher gas prices without paying correspondingly higher fees.
%The gas tokens works on taking advantage of storage refund concept in ethereum, to inspire smart contracts to delete storage variable, ethereum network provides refund when storage variable is deleted up to half of the contract transaction.
	% If a variable is changed from zero to a non-zero value, there is a gas fee  (writing into the storage is a very very expensive operation because it's altering the global state)
	% If a variable is changed from a non-zero value to zero, there is a gas refund
% A user can:
	% Mint tokens when GasPrice is low: change a variable from a zero value to non-zero.
	% Burn tokens when GasPrice is high: change a variable from non-zero to zero.

% As mentioned, the gas refunded from freeing some storage space is less than the initial gas cost of storing the data on-chain in the first place, so why would someone want to do this? Well, you can create these GasTokens (filling an array with ones or creating a bunch of contracts) at a time t with a low gas price (e.g. 2 Gwei) and later consume these GasTokens in a transaction that has a higher gas price. Hence, half of your gas used for the expensive transaction effectively costs you 2 Gwei, while the other half costs 50 Gwei. Even if you don’t get back as much gas as you spent creating these GasTokens, the difference in gas price can make it worth it.

% IMPORTANT: How is the mechanic? 1) when the Gas price is low, user calls the gastoken contract and asks it to store a bunch of words, this costs a lot of gas but because gas is cheap it will only cost a few cents for the user. 2) later in the future, where the user wants to register a domain and the gas price is so expensive, user will create a tx and within that he does 2 things: (i) register the domain she wants and (ii) ask gastoken contract to remove those words she's stored before.

%=================%
% Freeing up many storage slots in one transaction can quickly result in the refund counter surpassing half of the full transaction cost. In such a case or in case of self-destructs which can lead to a high refund counter as well we should evaluate how much of the refund can actually be used, as the refund can be at most half of the transaction cost. {The refund that has been accumulated cannot exceed half the gas used up during computation because miners need incentives to correctly execute your txs, and they should not pay Gas for your txs.} Thus freeing up storage slots or deleting contracts can make more sense in combination with other operations if possible (Gas Cost Analysis for Ethereum Smart Contracts: Zurich)
% Refund balance: It is the amount to be refunded to the sender account post transaction execution. Storage in Ethereum is quite expensive, so there is an SSTORE instruction in Ethereum that is used as a refund counter. The refund counter starts at zero (no refund state) and gets incremented every time the transaction or contract deletes something from the storage. Please note that this refund amount is different and in addition to the unused gas that gets refunded to the sender.
% In solidity, there are two commands which ensure that you get some gas refund back.
	% SUICIDE: This basically kills the smart contract. Doing so will get you back 24000 gas.
	%SSTORE: Storage deletion, which gets you back 15,000.
%=================%
% since minting new coin burns gas, you can just show up and give your gas token (and burn that) and ask the system to use that gas to mint new tokens. If you burn one gas token for one coin, it's pointless. But they could mint 1.5 coin using that gas token.

% Gas is a measure of computation, and the question we have to ask is is the price of computation stable?
% Hypothesis 1: Gas will look like exactly like Ether (cause it's Ethereum internal things),2 : looks like USD, because it should always cost 5 dollars to looked up an array (because the electricity cost is the same), 3- so it looks like electricity. There is a lot of reasons why things should converge, because of users incorrect mental model. They just can accept whatever their Ethereum client suggests, how this ethereum client gets those numbers? is it a real time or based on statistics? it can be gamed: sth like wash trade to make Gas cheaper (making it expensive is harder). What they could do is never broadcast it to the network so that nobody can put that in their block (causE THEy dont have the original tx to validate). So this way miner can keep those transactions in his private mempool and broad cast the whole block with that tx.



%=================IMPORTANT=================%
%=====Thoughts on GasToken used by Arbitrage for Font-running=======%
% arbitrage on the Ethereum network can use GasToken-like techniques. Arbitrage bots try to run the txs all the time and as soon as they see an opportunity to font-run some exchnage tx and make some money (cite front-running paper) so if they can lower their tx fees even a little bit, it extends the margins they can make (which txs are beneficial for them to front-run).


%========GasPrice Plots Explanation:=========%

The two spikes in the Fig~\ref{fig:Gas} correspond to (i) January 2018 when Cryptokitties \footnote{Cryptokitties website \url{https://www.cryptokitties.co/}} was launched for the first time and (ii) when the FCOIN \footnote{Fcoin website \url{https://www.fcoin.com}} was launched and required a lot of on-chain voting. Both these events have caused the GasPrice to go up as Ethereum users had to pay more Gas for their transactions to go through.

% Importnat note: If some companies (for example Fcoin) knows that they're gonna be sending lots of txs to the network that will definitely increase the Gas price in the future, they can store this gastoken ahead of time and benefit from using cheap Gas when the price is up.


%=========GasPlots========%
\begin{figure}[!htb]

	\centering
	\subfloat[Gas with respect to BTC and USD]{\includegraphics[width=0.45\textwidth]{figures/gas.pdf}\label{fig:gas1}}
	\hfill
	\subfloat[Gas with respect to ETH and Electricity price]{\includegraphics[width=0.45\textwidth]{figures/gasElectricity.pdf}\label{fig:gas2}}
	\caption {Ethereum average GasPrice chart. As mentioned in the Section~\ref{sec:GasInvs}, the two spikes in the chart represent specific events happened in certain dates which have increased the GasPrice.}
	\label{fig:Gas}

\end{figure}

%========================%




\textblue{ it would be nice to have a chart : Ether, Electricity (Global energy index), and Gas} \par
Our analysis in Section~\ref{sec:Intro} shows that cryptocurrencies exhibit less stable behavior compared to fiat currencies. Hence, we aim to analyze the price stability of \emph{Gas}. In Ethereum blockchain, every transaction contains a number of operations and there is a precise amount of gas unit associated with each.\footnote{http://ethdocs.org/en/latest/contracts-and-transactions/account-types-gas-and-transactions.html\#what-is-gas} Since Ether is volatile, this concept is introduced to have fixed cost for operations. Thus, even though the price of Ether changes, the amount of Ether corresponding to a unit of gas decreases; so that users pay the same price for the transactions on the Ethereum blockchain. Hence, the notion of gas is yet to be investigated to find out whether it exhibits behavior similar to fiat currencies or cryptocurrencies.

%========================%

%\begin{figure}[!htb]
%	\centering
%	\includegraphics[width=0.95\textwidth]{figures/eur.pdf}
%	\caption{\label{fig:eur} EUR with respect to BTC and gold.}
%\end{figure}
%========================%

%\begin{figure}[!htb]
%	\centering
%	\includegraphics[width=0.95\textwidth]{figures/eth.pdf}
%	\caption{\label{fig:eth} ETH with respect to BTC and gold.}
%\end{figure}
%========================%

%\begin{figure}[!htb]
%	\centering
%	\includegraphics[width=0.95\textwidth]{figures/gas.pdf}
%	\caption{\label{fig:gas} Gas with respect to BTC and gold.}
%\end{figure}
%========================%

%references to figures wrong now!

Figures~\ref{fig:eur}-~\ref{fig:gas} illustrate how the values of EUR, ETH and Gas change with respect to BTC and gold. EUR plot in Figure~\ref{fig:eur} tends to have more movements in the directions 1-5 and 3-7 suggesting that while the value of EUR stays the same according to one axis, it changes according to the other. For instance, between June 2018 and August 2018, the value of EUR increased with respect to BTC, while it stayed the same according to USD. This type of movement suggests that EUR-USD exchange rate is going under less change, whereas BTC-EUR rate is subject to high volatility. On the other hand, in the first half of 2017, while EUR retained its value against BTC, EUR to USD rate went under change.

ETH plot in Figure~\ref{fig:eth} illustrates more volatility against BTC and gold, as there are horizontal, vertical and diagonal changes. The fact that the points are spread in a large range of values indicates drastic changes in ETH price with respect to BTC and Gold.

Compared to Figure~\ref{fig:eur} and Figure~\ref{fig:eth}, gas plot (Figure~\ref{fig:gas}) has mostly diagonal changes, spread over a smaller range. There are less number of changes compared to ETH. Except from the changes between May 2018-July 2018 and January 2018-March 2018, the gas price changes in a smaller range. Even though there are fluctuations in the gas price, it can be inferred that gas price is less volatile than ETH. Also, it worths mentioning that gas is changing over a small scale in x-axis (USD), when compared to Ether's plot over the same axis in a larger interval.


%==========================================%
\begin{figure}[!htb]
	\centering
	\subfloat[ETH]{\includegraphics[width=0.45\textwidth]{figures/eth.pdf}\label{fig:cad}}
	\hfill
	\subfloat[XRP]{\includegraphics[width=0.45\textwidth]{figures/xrp.pdf}\label{fig:eur}}
	\caption{Volatility in cryptocurrencies}
	\label{fig:fiatandcrypto}
\end{figure}


%========================%

\begin{figure}[!htb]
	\centering
	\subfloat[CAD]{\includegraphics[width=0.45\textwidth]{figures/cad.pdf}\label{fig:cad}}
	\hfill
	\subfloat[EUR]{\includegraphics[width=0.45\textwidth]{figures/eur.pdf}\label{fig:eur}}
	\hfill
	\subfloat[Tether]{\includegraphics[width=0.45\textwidth]{figures/tether.pdf}\label{fig:tether}}
	\hfill
	\subfloat[BitUSD]{\includegraphics[width=0.45\textwidth]{figures/bitusd.pdf}\label{fig:bitusd}}
	\caption{Stability in two government-issued fiat currencies (CAD and EUR) and two stablecoin projects (Tether and BitUSD). Note that the x-axis is sized consistently across all four plots, with a \$0.30 USD spread. }
	\label{fig:fiatandcrypto}
\end{figure}
%========================%

%GRAPHS:
% Figure4: ETH,XRP - (BTC and USD): mostly diagonal movements. Do they move with BTC or not? Shows how cryptocurrencies are volatile. They don't move with BTC
%CAD, EUR,Tether, BitUsd: We expect stablecoins to be close to a vertical line. CAD and EUR are central bank issued "stablecoins" and compared to Tether and BitUsd (crypto stableoins), we can conclude that they are also fairly stable. (same x-axis range!)The graph seems like Tether's having drastic fluctuations. Note that the price change is around 10 cents. The change is accentuated due to the scale.
% TETHER: Does it artificially inflate BTC price? Do they issue un-backed tokens to reinforce BTC price, then sell BTC to fully back tokens?
%TETHER ATTACK: On October 15, 2018, tether, the market dominating stablecoin with a market cap of $2 billion, was attacked, breaking tether’s peg to USD, dropping its value by 7 percent but simultaneously driving up bitcoin and the whole crypto market by more than 10 percent.
%on October 15, 2018; where the price drastically surged from $6,376 at 12.31 UTC to $7,083 at 14.50 UTC. A downturn in price was seen approximately one hour after that to $6,821, before went stable at around $6,400 — $6,600. It was one of the one of the most unexpected fluctuations after a while. The first ‘stable coin’ tether (USDT) which supposedly pegged 1:1 to the U.S. dollar fell up to 15% at the same time as the price of bitcoin fluctuate wildly. The movement caused a loss of trust among traders, causing them to sell tether and invest in other cryptocurrencies. A massive selloff for tether against bitcoin was seen, resulting in a price rise in bitcoin.
%bitcoin and other crypto assets are perfectly negatively correlated with stablecoins.
%speculative attack

% Talk about Ripple surge in  December 2017 {https://www.forbes.com/sites/jessedamiani/2017/12/22/5-reasons-why-the-ripple-price-is-going-up-so-fast-will-the-xrp-surge-continue/#58337c997cbb}  reasons: altcoins gained importance, Rippe's strategy on partnership and customer acquisition, Ripple's partnersips in Asia, rumors that Coinbase will support xrp, "it's speed (4-second transactions) and low fees also make it appealing to general consumers"

\section{Conclusion and Discussion}
In this paper, we analyze the current state of stablecoins with the various options that have been so far proposed to achieve price stability. We also discuss various issues that stablecoins would address. According to the charts represented in the paper, gas is relatively stable in price, while Bitcoin and Ether show volatile behaviour. The reason could be the fact that how users interact with the interface to set the gas price when sending transactions to the Ethereum. By analyzing the properties of gas together with the existing methods to create stablecoin, we can later propose what properties stablecoins should attain.

%==========================================%
