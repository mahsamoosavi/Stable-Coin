% !TEX root = ../main.tex
%==========================================%

\section{Introduction}\label{Intro}
% cite {Trading and Exchanges - Larry Harris [PDF]} volatility chapter.
%book value and fundamental value

% It is difficult to define a stable coin because, the more we read, it captures a negative sentiment (what a coin ought not to be) more than what a coin ought to be. It ought not to be like Bitcoin, whose volatility is high.

% Stablecoins are a topic of recent interest. A lot of blogs and some academic articles have systemized them. We know that the reader of this paper is expecting a chart of coins and what mechanisms they use: we do provide that, but we do not consider that the core contribution of the paper at all. Instead we really dug into finance to deeply understand the core methods and mechanisms that are employed for reducing volatility of a coin. We feel our paper is closer to a tutorial on stablecoins where we try to eliminate all the jargon (which is often misused and imprecisely applied) to explain the concepts, as well as drawing out appropriate foundations from finance.

% Example: if we described the approach of stable coin in two simple words: ``currency board'' and you need no further explanation (just, perhaps, some details on the parameters), this paper is not for you. If you have heard the term and have a vague sense of how it works, but couldn't really explain it or why it achieves stability, this paper is for you. If you have never heard the term, this paper is also for you.

% Two basic concepts: (1) a tokenization of a fiat currency (say a digital dollar) or any asset or asset portfolio of assets. This is an old idea: liberty reserve, eGold. (2) a low-volatility coin that is tied to any existing asset. For example, with USD as the baseline, Euros are much more stable than Bitcoin. It is not because Euros are backed by USD or have any direct reference to the USD. It is because USD and Euros use the same model, a central bank management, with similar inputs (the interest rates their customers, commercial banks, use when lending cash to each other to meet various legal requirements that help ensure the banks will have enough cash on hand to server their customers and can withstand some of their investments going bad without going bankrupt).

% Most of the coins are mostly focusing on the supporting infrastructure, maybe we also have to talk about the supporting infrastructure of the stablecoin
% Talk about the crypto custodianship somewhere in the paper
%==========================================%

\subsection{Motivation} %need a better name
\textblue{In this section we talk about volatility and what does stability mean, and we make argument that we're not economist and everything is explained in our own language}
Cryptocurrencies have gained a wide application after Bitcoin was first introduced in Satoshi Nakamoto’s (pseudonymous) 2008 whitepaper~\cite{nakamoto2008bitcoin}. For Bitcoin and any other cyptocurrency to function as money, they need to fulfill a set of properties that determine the strength and adoption of them \ie they are expected to serve as a medium of exchange, a unit of account, and a store of value. However, due to high fluctuations in their prices, majority of the cryptocurrencies do not meet these properties and hence they cannot be adopted as money~\cite{overview}.

Having said that, the volatile nature of cryptocurrencies (\eg Ether, Bitcoin) has raised the interest into what is known as stablecoin. Stablecoins (\ie cryptocurrencies with stable price) ensure that the fluctuation in the value remains low. Figure~\ref{fig:btcandfiat} illustrates the volatility of Bitcoin's value, when compared to fiat currencies, and the change of values of EUR, GBP, CAD, and BTC with respect to USD over time. Monthly values between January 2016 and November 2018 are shown. According the figures, while fiat currencies show stable behaviour, Bitcoin's value changes drastically over time, which makes it a non-stable cryptocurrency.

%========================%
\begin{figure}[!htb]
	\centering
	\includegraphics[width=0.75\textwidth]{figures/allCurrencies.pdf}
	\caption{\label{fig:btcandfiat}Comparison among fiat currencies and Bitcoin: The values are retrieved daily between  01-01-2016 and  01-01-2019.}
\end{figure}
%========================%


%========================%

\begin{figure}[!htb]
	\centering
	\subfloat[GBP with respect to EUR and USD]{\includegraphics[width=0.75\textwidth]{figures/gbpBrexit.pdf}\label{fig:btc}}
	\hfill
	\subfloat[Legend]{\includegraphics[width=0.25\textwidth]{figures/compass}\label{fig:legend}}
	\caption{BTC}
	\label{fig:Comparison}
\end{figure}
%========================%


Figure~\ref{fig:btc} shows the change of value of Bitcoin with respect to USD and gold. The value changes happen in directions 2 and 6 (Figure~\ref{fig:legend}). The part of plot in direction 2 means that Bitcoin is gaining value against gold and USD. This can also be interpreted as both gold and USD are losing value relative to Bitcoin. %However, the first explanation is more likely to be the case, as only Bitcoin's value is changing compared to two values (USD and gold) changing.
Also, the fact that the plot is located on the diagonal shows that Bitcoin gains/loses value against both USD and Gold at the same time. This indicates that the changes in the values of USD and gold are highly correlated.

Considering these facts, there is a desire to design stablecoins with the stable nature of the fiat currencies together with the decentralized nature of the blockchain which is the underlying technology of cryptocurrencies.

%==========================================%

\section{Preliminaries}

\subsection{Valuation}
% Every thing has two prices

%How do we assign value to something? how do we establish what something is worth?  The value is usually given in some units of account (\eg CAD).

%===Explaining the crossing===%

%When we establish a value for something, we do not really claim what the value is and what it worths, instead, we establish a value of its exchange. For example, today Alice is willing to pay 500 CAD for Bob's phone, whereas tomorrow, she will pay 600 CAD. What happened in this case? It could be (i) the value of money (CAD) going down or (ii) the value of good (phone) going up. Thus, in order to establish how much value something has, we should think of it in terms of being an exchange rate (the exchange rate between that asset and some other valuable assets). In other words, if Bob is willing to offer his phone for 1000 CAD, does it mean that it worths 1000 CAD? What we know about is that the value of Bob's phone is somewhere between 500 CAD (which Alice is willing to bid) and 1000 CAD (which Bob is willing to accept/ask). Therefore, the value of an object is not a single number and we can think of it as an interval between a least amount somebody wishes to sell for and the most amount somebody wishes to pay for that object, this interval is called bid-ask. If an ask comes below a bid, then the transaction happens and that amount is recorded as the final price, we call this \textit{crossing}.

%What actually happens when they cross? Assume that they cross in such that Bob wants to pay (bid) 600 CAD for Alice's phone while Alice wants to sell (ask) it for 500 CAD. What happens in the real world is (i) either one party goes first and advertises a bid or an ask and when it crosses with other party's complimentary bid or ask, she accepts it. (ii) both parties advertise and they do not know about each other's price, so when they cross, the market decides (usually they pick the midpoint between the two).

%When there is a cross, you have a price that is used for selling the object, that price is a last transaction price. So three values are actually important: 1) What is the most somebody is willing to pay for the object, 2) What's the least somebody is willing to accept for it, and 3) what was the last price this object is sold out. The value we see in the newspaper is always the last price, although it does not represent how much the object worths. In fact, we cannot actually say that and the best way to say that is when the bid-ask interval is really small. Assume that the interval is only 1 cent difference, in this case we can make sure what is the value and we are sure down to a cent. When the market changes so fast, the last price gives a lot of information on how much something worths, however, the last price that belongs to 30 years ago does not reveal a lot of information.

\subsection{Exchange}

\clearpage
\subsection{No-Seignorage Theory of Money}

To explain Bitcoin's exchange rate with fiat currencies, an oft-repeated theory has emerged that attributes Bitcoin's value to the hydro consumed by blockchain mining. While imprecise, the theory suggests that if a valuable resource $x$ is consumed to produce $y$, the value of $x$ is imparted into $y$. Setting aside the nuance that the hydro contributed to the Bitcoin system only indirectly produces new coins (it produces blocks, and blocks produce coins only for now), there is no economic principle underlying this transfer of value.

If Alice goes to Peter Luger's in Brooklyn, consumes a \$100 ribeye, and mints a literal shitcoin out of the result --- is that coin worth \$100 because it is ``backed'' by \$100 worth of steak?

\subsection{plot explanations}


%=======Plot Legend Explanation===========%
\begin{table}[t]
\centering
\begin{tabular}{|l|l|}
\hline
\textbf{Direction} & \textbf{Interpretation}   \\ \hline

         1/5                & $Y$ is losing (1) / gaining (5) value \\ \hline
         2/6                & Plotted asset is gaining (2) / losing (6) value \\ \hline
         3/7                & $X$ is losing (3) / gaining (7) value \\ \hline
         4/8                & \multicolumn{1}{p{12cm}|}{Plotted asset is gaining (4) / losing (8) value against $X$, while losing (4) / gaining (8) value against $Y$} \\ \hline

\end{tabular}
\vspace{1em}
\caption{\footnotesize{The interpretation of the plots.}\label{tab:plotlegend}}
\end{table}
%===================%



%==========================================%

\section{Systemization of stablecoins and Justification}

% Dollar tokenization vs. low volatitility

% !TEX root = ../main.tex


%-------------------Fancy Table ----------------------%

% = = = Rotated Table Entry \headrow

%\usepackage{adjustbox}
%\newcommand{\headrow}[1]{\multicolumn{1}{c}{\adjustbox{angle=45,lap=\width-0.5em}{#1}}}

% = = = Table bullets: \full and \prt (full and part)

\newcommand{\full}{$\bullet$}
\newcommand{\prt}{$\circ$}
% ------------------------------------------------------------------------------------------------------------------------------------------------%
%Stablecoin projects%

\definecolor{UnitedNationBlue}{rgb}{0.30,0.53,1}
\definecolor{LightSteelBlue}{rgb}{0.69,0.77,0.87}
\definecolor{LightGrey}{rgb}{0.83,0.83,0.83}


\begin{table}[h!]
\centering

\begin{tabular}{|l|l|l|l|l|}

\hline
\rowcolor{lightgray}
\textbf{Category} & \textbf{Sub-Category} & \textbf{Examples} & Rank \\  \hline

\multirow{16}{*}{Backed}		& \multirow{8}{*}{Directly-Backed \& Redeemable}					& \textbf{USDC} & 20 \\ \cline{3-4}
						&														& TrueUSD & 26 \\ \cline{3-4}	
						&														& Paxos & 38 \\ \cline{3-4}												
						&														& Gemini Dollar & 52 \\ \cline{3-4}
						&														& StableUSD (USDS) & 685 \\ \cline{3-4}
						&														& Stronghold USD & 891 \\ \cline{3-4}
						&														& Petro & 1210 \\ \cline{3-4}
						&														& \multicolumn{1}{p{5cm}|}{Ekon, WBTC, emparta} & $\perp$ \\ \cline{2-4}
						& \multirow{6}{*}{Directly-Backed}  								& \textbf{Tether} & 6 \\ \cline{3-4}
						&														& EURSToken & 95 \\ \cline{3-4}
						&														& BitCNY & 304 \\ \cline{3-4}
						&														& Terracoin & 1280 \\ \cline{3-4}
						&														& Saga & 1495 \\  \cline{3-4}
						&														& \multicolumn{1}{p{5cm}|}{GJY, Novatti AUD, UPUSD} & $\perp$ \\ \cline{2-4} 						
						%& \multirow{1}{*}{Non-Redeemable and Indirectly-Backed}  			&  &  \\ \cline{2-4}
						& \multirow{2}{*}{Indirectly-Backed}								& \textbf{Dai} & 57 \\ \cline{3-4}
                                                &														& BitUSD & 398 \\  \cline{1-4}
 
                                                                                                                 
                                                                                                                 
                                                                                                                 
\multirow{5}{*}{Intervention}                                                           
						& \multirow{1}{*}{Market} 										& Nubits & 892 \\ \cline{2-4}
						& \multirow{1}{*}{Algorithmic with subjective external information}  		& Nomin* & $\perp$  \\ \cline{2-4}
						& \multirow{2}{*}{Algorithmic with objective external information}  		& CarbonUSD* & 1262 \\ \cline{3-4}
						&														& Ampleforth* & $\perp$ \\ \cline{2-4}
						& \multirow{1}{*}{Algorithmic with only internal information} 			& Basis & $\perp$ \\ \hline

\end{tabular}
\vspace{1em}
\caption{\footnotesize{Systemization the current stablecoin projects} on January 11, 2019. The Projects that are in bold are discussed in this paper.}
\label{tab:stablecoins}
\vspace{3em}
\end{table}
% ------------------------------------------------------------------------------------------------------------------------------------------------%

% Jeremy: How do we make a stable coin.
% Jeremy: Table here

\section{Backed}
% issue: the stablecoin is valid as long as the holder trusts the company/person who keeps the reserve, once the trust is gone, the stablecoin worths nothing

\subsection{Redeemable and Directly-Backed.}
% Description of how it works: Deposit in bank, etc.
% Regular audits are needed to ensure that the stablecoin is indeed fully collateralized. Although it's redeemable, a few people can redeem it and that itself shows that they have enough reserve, BUT what if nobody wants to redeem it?
% Justification that bids will never exceed $1: an arbitrageur will pay $1 to mint 1 XSC and the 1 XSC for the bid
% Justification that offers will be less than $1: an arbitrageur will purchase the 1 XSC and redeem it for $1
% Risk: not redeemable -> redemption is not 100%, the coin will be offered at less than $1. E.g., offers of $0.50, says 50% it can't be redeemed.
% Examples: Gemini
% Mahsa: Should we talk about these two recent incidents in this section?
% According to the recent blogpost in coindesk (https://www.coindesk.com/winklevoss-crypto-gemini-gusd-stablecoin-redemption) Gemini has closed the account of some users and in some cases do not let redemption (some KYC issues) +     Paxos has has the same issue, although they let the user redeem their assets and then closed their account (https://www.ccn.com/paxos-standard-hassling-ethereum-traders-trying-to-redeem-stablecoin-pax-for-dollars)


\subsection{Non-Redeemable and Directly-Backed.}
% Description of how it works. Deposit in bank, etc.
% Regular audits are needed to ensure that the stablecoin is indeed fully collateralized
% Justification that bids will never exceed $1:
% Justification that offers will be less than $1
% Examples: Tether
% Remarks:
%

\subsection{Non-Redeemable and indirectly-Backed.}
% Description of how it works.
% Justification that bids will never exceed $1
% Justification that offers will be less than $1
% Examples:
% Remarks: Oracles


\subsection{Redeemable and indirectly-Backed.}
% Description of how it works.
% Justification that bids will never exceed $1
% Justification that offers will be less than $1
% Examples:



\section{Intervention-based} % it used to be Algorithmic

% Control system: input (metrics), make a decision, implement the intervention
% We cannot prove that something here works or not. It is all heuristics. We know this because central banks themselves operate on heuristics that change every couple of decades. Even if we cannot say something does work, we can point out a few suggestions that it will not work. Game-able inputs, inputs/interventions that have not worked for currency.

\subsection{Currency Board}
% Description of how it works.
% Justification that bids will never exceed $1
% Justification that offers will be less than $1
% Examples:

\subsection{Algorithmic with subjective external information}
% Description of how it works.
% Justification that bids will never exceed $1
% Justification that offers will be less than $1
% Examples: RSCoin

\subsection{Algorithmic with objective external information}
% Description of how it works.
% Justification that bids will never exceed $1
% Justification that offers will be less than $1
% Examples:

\subsection{Algorithmic with only internal information}
% Description of how it works.
% Justification that bids will never exceed $1
% Justification that offers will be less than $1
% Examples:

We performed a search query on \texttt{coindesk.com} and found the following projects which are mostly used in every articles published about stablecoins until January 11, 2019. \footnote{https://www.coindesk.com/} Table~\ref{tab:stablecoins} represents these projects and the methods they apply to achieve stability. (we should mention the search term that we used to extract our resources from coindesk)


\textbf{Distribution of articles speaking about stablecoin:} 2014: 2, 2017: 4, 2018: 112, 2019 (up to Jan 11): 4.

%================Notes on unranked projects:==========================% 

%CarbonUSD: 
%t hybridizing both fiat backed and algorithmic stablecoins, (hybrid fiat-algorithmic approach)
% 1-1 backed with USD
% unlike purely fiat-backed stablecoin, CarbonUSD's novel meta-token structure enables a seamless future transition to an algorithmic stablecoin model once it reaches sufficient scale as a fully fiat-backed token.
% meta-token smart contract is a key innovation that enables carbonusd to eventually transition from full fiat-collateralization without distrupting its liquidity network effects on exchanges and while maintaining the highest standards for regulatory compliance.
% HOW? CarbonUSD is a basket of whitelisted tokens. A token that is whitelisted may be used to create new CarbonUSD, serving as its collateral. Initially, only one token will be whitelisted, a stablecoin that is 1-1 backed with USdollars in a trust account.
% The first whitelisted token, WT0, is a compliant fiat-backed stablecoin where users can deposit and withdrawal real USD. Governing members of the whitelist can decide when CarbonUSD has enough liquidity to safely switch off from full fiat- collateralization.
% they also have regular audits (unlike Tether)
% Redeemable
% Offchain (fiat-backed, crypto-backed) stablecoins are effective as bootstrapping trust and NOT maintaining it scale. HOWEVERM, Onchain (algorithmic) stablecoins are effective at maintaining trust at scale BUT not at bootstrapping it.
%=================% 

%==========================================%

\section{Investigation of gas volatility}

% Gas: non-redeemable backed by Ether. like putting Ether in the contract and get USD digital token. why non-redeemable? because you burn Gas to get your digital tokens. But in order to make it redeemable, there is a way in Ethereum to do that, here is the GasToken that uses this strategy.
%since minting new coin burns gas, you can just show up and give your gas token (and burn that) and ask the system to use that gas to mint new tokens. If you burn one gas token for one coin, it's pointless. But they could mint 1.5 coin using that gas token.
% Gas is not a currency so you cannot buy things with it BUT you can turn it into a currency : having a smart contract and swap it with other tokens
% Gas is actually stable with respect to others, the reason it's not obvious is human error and their wrong mental model
% Gas is not pegged to anything
% Gas is free floating but it’s driven (governed) by market forces (auctioning off)
% there is an upper band and lower band for Gas
% Gas is a measure of computation, and the question we have to ask is is the price of computation stable?
% Hypothesis 1: Gas will look like exactly like Ether (cause it's Ethereum internal things),2 : looks like USD, because it should always cost 5 dollars to looked up an array (because the electricity cost is the same), 3- so it looks like electricity. There is a lot of reasons why things should converge, because of users incorrect mental model. They just can accept whatever their Ethereum client suggests, how this ethereum client gets those numbers? is it a real time or based on statistics? it can be gamed: sth like wash trade to make Gas cheaper (making it expensive is harder). What they could do is never broadcast it to the network so that nobody can put that in their block (causE THEy dont have the original tx to validate). So this way miner can keep those transactions in his private mempool and broad cast the whole block with that tx.



\textblue{ it would be nice to have a chart : Ether, Electricity (Global energy index), and Gas} \par
Our analysis in Section~\ref{Intro} shows that cryptocurrencies exhibit less stable behavior compared to fiat currencies. Hence, we aim to analyze the price stability of \emph{Gas}. In Ethereum blockchain, every transaction contains a number of operations and there is a precise amount of gas unit associated with each.\footnote{http://ethdocs.org/en/latest/contracts-and-transactions/account-types-gas-and-transactions.html\#what-is-gas} Since Ether is volatile, this concept is introduced to have fixed cost for operations. Thus, even though the price of Ether changes, the amount of Ether corresponding to a unit of gas decreases; so that users pay the same price for the transactions on the Ethereum blockchain. Hence, the notion of gas is yet to be investigated to find out whether it exhibits behavior similar to fiat currencies or cryptocurrencies.

%========================%

%\begin{figure}[!htb]
%	\centering
%	\includegraphics[width=0.95\textwidth]{figures/eur.pdf}
%	\caption{\label{fig:eur} EUR with respect to BTC and gold.}
%\end{figure}
%========================%

%\begin{figure}[!htb]
%	\centering
%	\includegraphics[width=0.95\textwidth]{figures/eth.pdf}
%	\caption{\label{fig:eth} ETH with respect to BTC and gold.}
%\end{figure}
%========================%

%\begin{figure}[!htb]
%	\centering
%	\includegraphics[width=0.95\textwidth]{figures/gas.pdf}
%	\caption{\label{fig:gas} Gas with respect to BTC and gold.}
%\end{figure}
%========================%

%references to figures wrong now!

Figures~\ref{fig:eur}-~\ref{fig:gas} illustrate how the values of EUR, ETH and Gas change with respect to BTC and gold. EUR plot in Figure~\ref{fig:eur} tends to have more movements in the directions 1-5 and 3-7 suggesting that while the value of EUR stays the same according to one axis, it changes according to the other. For instance, between June 2018 and August 2018, the value of EUR increased with respect to BTC, while it stayed the same according to USD. This type of movement suggests that EUR-USD exchange rate is going under less change, whereas BTC-EUR rate is subject to high volatility. On the other hand, in the first half of 2017, while EUR retained its value against BTC, EUR to USD rate went under change.

ETH plot in Figure~\ref{fig:eth} illustrates more volatility against BTC and gold, as there are horizontal, vertical and diagonal changes. The fact that the points are spread in a large range of values indicates drastic changes in ETH price with respect to BTC and Gold.

Compared to Figure~\ref{fig:eur} and Figure~\ref{fig:eth}, gas plot (Figure~\ref{fig:gas}) has mostly diagonal changes, spread over a smaller range. There are less number of changes compared to ETH. Except from the changes between May 2018-July 2018 and January 2018-March 2018, the gas price changes in a smaller range. Even though there are fluctuations in the gas price, it can be inferred that gas price is less volatile than ETH. Also, it worths mentioning that gas is changing over a small scale in x-axis (USD), when compared to Ether's plot over the same axis in a larger interval.
%========================%

\begin{figure}[!htb]
	\centering
	\subfloat[CAD]{\includegraphics[width=0.45\textwidth]{figures/cad.pdf}\label{fig:cad}}
	\hfill
	\subfloat[EUR]{\includegraphics[width=0.45\textwidth]{figures/eur.pdf}\label{fig:eur}}
	\hfill
	\subfloat[Tether]{\includegraphics[width=0.45\textwidth]{figures/tether.pdf}\label{fig:tether}}
	\hfill
	\subfloat[BitUSD]{\includegraphics[width=0.45\textwidth]{figures/bitusd.pdf}\label{fig:bitusd}}
	\caption{CAPTION}
	\label{fig:fiatandcrypto}
\end{figure}
%========================%


%========================%
\begin{figure}[!htb]
	\centering
	\subfloat[ETH]{\includegraphics[width=0.45\textwidth]{figures/eth.pdf}\label{fig:cad}}
	\hfill
	\subfloat[XRP]{\includegraphics[width=0.45\textwidth]{figures/xrp.pdf}\label{fig:eur}}
	\caption{CAPTION}
	\label{fig:fiatandcrypto}
\end{figure}
%========================%
\begin{figure}[!htb]
	\centering
	\subfloat[Gas with respect to BTC and USD]{\includegraphics[width=0.45\textwidth]{figures/gas.pdf}\label{fig:cad}}
	\hfill
	\subfloat[Gas with respect to ETH and Electricity price]{\includegraphics[width=0.45\textwidth]{figures/gasElectricity.pdf}\label{fig:eur}}
	\caption{CAPTION}
	\label{fig:fiatandcrypto}
\end{figure}

%==========================================%
%GRAPHS:
%About figure1: EUR,CAD and GBP lines are nearly parallel. GBP and EUR come closer after Brexit. Even such an important economical event didn't cause a high volatility compared to BTC price which is more extreme.
%All prices between 01-2017 and 11-2018
%GBP -(EUR and USD) chart "The pound dropped to its lowest level against the euro in 2017 after Mario Draghi, head of the European Central Bank, pushed the eurozone currency higher by saying he could start tightening monetary policy." September 2017 GBP loses value against both USD end EUR
% ETH,XRP - (BTC and USD): mostly diagonal movements. Do they move with BTC or not?
%CAD, EUR,Tether, BitUsd: We expect stablecoins to be close to a vertical line. CAD and EUR are central bank issued "stablecoins" and compared to Tether and BitUsd (crypto stableoins), we can conclude that they are also fairly stable. (same x-axis range!)
% TETHER: Does it artificially inflate BTC price? Do they issue un-backed tokens to reinforce BTC price, then sell BTC to fully back tokens?
%TETHER ATTACK: On October 15, 2018, tether, the market dominating stablecoin with a market cap of $2 billion, was attacked, breaking tether’s peg to USD, dropping its value by 7 percent but simultaneously driving up bitcoin and the whole crypto market by more than 10 percent.
%on October 15, 2018; where the price drastically surged from $6,376 at 12.31 UTC to $7,083 at 14.50 UTC. A downturn in price was seen approximately one hour after that to $6,821, before went stable at around $6,400 — $6,600. It was one of the one of the most unexpected fluctuations after a while. The first ‘stable coin’ tether (USDT) which supposedly pegged 1:1 to the U.S. dollar fell up to 15% at the same time as the price of bitcoin fluctuate wildly. The movement caused a loss of trust among traders, causing them to sell tether and invest in other cryptocurrencies. A massive selloff for tether against bitcoin was seen, resulting in a price rise in bitcoin.
%bitcoin and other crypto assets are perfectly negatively correlated with stablecoins.
%speculative attack

% Talk about Ripple surge in  December 2017 {https://www.forbes.com/sites/jessedamiani/2017/12/22/5-reasons-why-the-ripple-price-is-going-up-so-fast-will-the-xrp-surge-continue/#58337c997cbb}

\section{Conclusion and Discussion}
In this paper, we analyze the current state of stablecoins with the various options that have been so far proposed to achieve price stability. We also discuss various issues that stablecoins would address. According to the charts represented in the paper, gas is relatively stable in price, while Bitcoin and Ether show volatile behavior. The reason could be the fact that how users interact with the interface to set the gas price when sending transactions to the Ethereum. By analyzing the properties of gas together with the existing methods to create stablecoin, we can later propose what properties stablecoins should attain.

%==========================================%



