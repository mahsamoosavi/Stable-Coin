% !TEX root = ../main.tex
\section{Introduction}
Very quick introduction to blockchain/crypto
What is stability vs volatility.
About coin supply and volatility~\cite{sams2015note}.
Short-term stability is important for transactions and long-term stability is important for holding`\cite{forbes}.


currencies are judged by their ability to serve three functions???as a medium of exchange, a unit of account, and a store of value. To successfully serve these functions, a currency must be easily and broadly usable (necessary for being a medium of exchange) and demonstrate at least some level of value stability (necessary for acting as a unit of account and store of value).

%For the paper let's have a section under preliminaries and talk about what is money?where does it come from?
%useful articles for this topic:
%https://www.imf.org/external/pubs/ft/fandd/2012/09/basics.htm
%https://www.imf.org/external/pubs/ft/fandd/2010/03/basics.htm

\section{The current state of the stable coins} % Didem :

%different ways and issues with each. %~\cite{euromoney}
Stablecoins has a market value of \$3 billion and this corresponds to the 1.5\% of the total market value of the cryptoassets~\cite{report}. Each proposing different properties, stablecoins can be categorized into three groups based on the way they achieve stability: fiat-collateralized, crypto-collateralized, and non-collateralized.

~\textbf{1) Fiat-collateralized stablecoins:} These type of stable coins are backed by fiat currency and backing by USD is one of the most common types. Generally, there is a 1:1 peg between the fiat currency and the stablecoin that indicates a convergence between their values~\cite{linkedin}. USD being one of the most common choices for the fiat currecncy to back the stablecoin, IBM states that they are also interested in projects that use other national fiat currencies, as they will be helpful for IBM's blockchain integration~\cite{cointelegraph}.


Tether and TrueUSD are USD are prominent examples of USD pegged tokens. Some projects like Digix Gold Token prefer to use gold to back their stablecoin, as gold has a relatively slow increase in its value compared to fiat currencies. %Reference??

~\textit{Discussion about centralization:} Backing up with fiat currency means that there is a need for third party. The amount of money to back the stablecoin up should be held in an account~\cite{techrev}. The involvement of a third party causes controversy in the community, as the third party can just deny giving money to the users. Tether explains this point as follows~\cite{cryptoinsider}:

\begin{quote}
"Redemptions will not be unreasonably denied, but we reserve the right to selectively deny redemption and creation of Tethers on a case-by-case basis."
\end{quote}

%Globcoin: less reliance on US dollars but still centralized.~\cite{cryptoinsider}

~\textbf{2) Crypto-collateralized stablecoins:} These type of stablecoins uses other cryptocurrencies as a back up value rather than a fiat currency. Over-collateralization is needed this case as the underlying cryptocurrency is also volatile~\cite{linkedin}. MakerDAO and Reserve use this approach. Reserve utilizes a smart contract to back the stablecoin with another cryptocurrency~\cite{cointelegraph}.

If there is a black swan event~\footnote{A black swan event is characterized as being unexpected, random and having significant effects to the current situation, hence it is hard to predict.(https://www.investopedia.com/terms/b/blackswan.asp)} where the underlying currency loses its value and does not worth anything, the stablecoin also loses its value~\cite{coinsexplained}.  Due to the over over-collateralization in crypto-collateralized stablecoins the loss of value will be drastic.

In this case the loss-exposure would even be amplified for the stablecoin owners because of the over-collateralization. This is also why some experts are strongly discouraging this approach.


~\textbf{3) Non-collateralized stablecoins:} Unlike the previous types of stable coins, these aren't back by fiat currencies of another cryptocurrency. The stability is achieved algorithmically~\cite{linkedin}. This provides better scalability~\cite{report}. Basis is one of the first projects that use this approach.

Basis and Carbon use the dual-token model~\cite{cryptoinsider}. There is dynamic adjustment of the existing supply of the stablecoin. While one token is stable, the other is used to achieve the stability of the value.

%The dual-token model — stablecoins and shares One token is stable, and the other is used to whip it back onto the narrow track. Roughly, the idea is that supply of the stable token is dynamically increased and decreased. ex: Basis, Carbon. Then the dual-token model has two flavours: the kind where the stable token is created by locking up other crypto (e.g. DAI and Havven), and the kind where it is not (e.g. Basis, Carbon). ~\cite{cryptoinsider}  DAI Havven crypto collat?

Another approach is the seigniorage shares method\cite{overview}. Here, the smart contract automatically adjusts the supply based on the algorithm to achieve stability in the value.

Basis is intended to peg at roughly one-to-one against the dollar. If it gains acceptance as a popular medium of exchange in the crypto world and increases in value to, say, \$1.10, the system will print more Basis tokens to increase supply and so reduce the price.  If the price falls below \$1, the code will issue bonds worth one basis token each, use the proceeds to buy existing Basis tokens to reduce supply and so bid the price back up, later repaying bondholders when Basis tokens trade above par.  It is a complicated, seigniorage based system.
https://www.euromoney.com/article/b1bbk5rb8gp227/forget-bitcoin-stablecoins-will-change-how-money-works?copyrightInfo=true


\section{Problems that stable coin addresses}
As mentioned in the ..., currencies can serve as a store of value, a unit of account, and a medium of exchange~\cite{smithin2002money}. In order to serves as a unit of account and store of value, currencies have to denote a minimum level of value stability. In this regard, stable coins are proposed to provide store of value and unit of account functionalities, due to their non-fluctuating value in terms of fiat currencies or any other alternative. In addition, they purport to solve a group of critical issues that were introduced when using cryptocurrencies introduce. In this section, we discuss these issues.

\subsection{Cryptocurrencies as Medium of Exchange}
from the blockchain paper:
At present, any business would take a significant risk accepting cryptocurrencies as a medium of exchange
due to the significant volatility of this asset class. Stablecoins hold the potential to help unlock the use of
cryptocurrencies for day-to-day payments for businesses and commerce as price stability is a key missing
element for the adoption of cryptocurrencies by merchants and retailers all over the world.
Companies need a degree of certainty about their short-term cash reserves and revenues. Transacting
in ether or bitcoin would make the role of a treasurer a difficult task as the business?s runway (how long
the company can survive if income and expenses stay constant) could adversely shift in an instant due to
unfavorable market swings.

\subsection{Cryptocurrencies as Unit of Account}
from the blockchain paper:
The unit of account is the measure by which goods and services are priced and a necessary feature for a
given asset to become ?money?. In the US, retailers price goods in USD, employees are paid in USD by their
employers, profits/losses and assets/liabilities are denominated in USD. There is currently no agreement
regarding the intrinsic value (and future value) of a given cryptocurrency, meaning accepting bitcoin as a
?unit? is therefore problematic.
Stablecoins can be pegged to established units of accounts in their respective countries and can thus
become a digital representation of the unit of account (so long as the peg is maintained). Given their
emphasis on price stabilty and the ability to peg stablecoins to inflation, stablecoins also arguably have a
greater chance of becoming an independent unit of account in the longer-term.

\subsection{Cryptocurrencies as Store of Value}
from the blockchain paper:
A store of value is a commodity, asset, or money that retains its purchasing power or value into the future.
Some view cryptoassets including bitcoin as too volatile to be commonly accepted as a store of value.
Some companies need to hedge themselves over the long-term. For example, miners are currently highly
exposed to the price of the cryptoasset they receive in return for computing resources. A stable reserve of
liquid assets is needed to cover one-off additional fixed costs (such as purchasing hardware) and on-going
variable costs (such as electricity).
In the current crypto ecosystem, volatility risk is currently being highlighted in fundraising via Initial Coin
Offerings (?ICOs?). Projects generally raise a given amount of ether to allocate resources in order to deliver
on their promises. Due to the high level of friction associated with converting cryptoassets into fiat,
founding teams tend to hold most of their funds in ether. In a bear market associated with falling prices
like the present one, management would have to meet investor expectations while suddenly having less
capital at their disposal. Stablecoins could thus help founding teams of ICO projects manage their funding
more saftely over the long-term.

\subsection{dApps}
from the blockchain paper:
In the web 3.0 stack, decentralized applications (?dApps?) are being built on top of infrastructure protocol
layers. Many of those applications will likely rely on price stable cryptocurrencies to distribute value.
Stablecoins should accelerate the shift from token speculation to usage in dApps as users won?t be
incentivised to hold (or sell) the token in anticipation of future price appreciation (or depreciation). This
should in turn increase the token velocity and fulfil the potential of decentralised networks.
dApps are the channel through which stablecoins are likely to be brought to the masses in the foreseeable
future. For example, MakerDAO and Dether recently partnered to bring Dai to mobile ATMs.
Finally, ERC20 stablecoins can be held and transferred by anyone who already has an Ethereum wallet, and nearly half of all stablecoin projects (48\%) are running on Ethereum. Provided that Ethereum is a successful underlying infrastructure protocol for dApps, ERC20 stablecoins should be adopted faster and benefit from the Ethereum vibrant ecosystem.

\subsection{Lending with Cryptocurrencies}
Despite quite a few blockchain applications in financial technologies, there has been little deployment of lending. One of the main challenges with the lending that makes it difficult to be deployed on the blockchains is the monetary instability observed in the the existing cryptocurrencies~\cite{okoyetoward}. This volatility has led the existing cryptocurrencies to be used more as speculative investments instead of serving as store of value and unit of account. In a lending situation with volatile currencies, where their value is being depreciated or appreciated over time, the cash taker will eventually owe more than what he has borrowed or the vice versa. Therefore, the volatility in the value of cryptocurrencies cause serious concerns and difficulties both for cash takers and cash providers~\cite{okoyetoward}. In the contrast, lending perfectly works if a loan is done with a stable cryptocurrency, whose value remains stable over the time.

\subsection{Government Surveillance}

Governments have recently started examining  the idea of issuing a central bank-issued digital currency (CBDC)~\cite{barrdear2016macroeconomics}. These national digital currencies are regulated by federal regulators instead of serving in a form of decentralized currencies-- where only the owner has the sole control and ownership of the assets. CBDCs are not backed by any tangible assets and are issued by central banks so that they could keep their monetary policy while adopting the trend of digital assets. However, an oppressive government may abuse these assets to manipulate the markets or to limit the ownership of these assets to  a special group of people (\eg citizens of that specific country). However, since stable coins are not backed by any central party and financial regulator, users are assured about the stability as well as having easier access no matter where they live and/or come from.

\subsection{Remittance}
from the blockchain paper:
Stablecoins eliminate price volatility risk as crypto payments are being processed. To stay relevant in this
context, transactions would have to be confirmed rapidly (ideally in a matter of seconds) to provide a good
user experience and a noteworthy improvement compared to transfers relying on the current underlying
banking infrastructure (international banking transfers currently take up to three days).

% nation issued digital assets can be
%
%After all, fiat currencies are not backed by any tangible assets. You can?t return the currency to the government in exchange for a bar of gold or silver, a can of beans, a pack of cigarettes, or any other items that might have value to you. Fiat currencies are backed by the full faith and credit of the government that issued them and nothing more. If you want gold, silver, beans, or smokes you need to exchange your fiat currency with a person or entity that possesses the item that you want.
%
%Read more: Why Governments Are Afraid of Bitcoin | Investopedia https://www.investopedia.com/articles/forex/042015/why-governments-are-afraid-bitcoin.asp#ixzz5VLsy6DKS
%Follow us: Investopedia on Facebook
%
%
%The reason is tha due tokeep their monetary poli and power.
%
%
%
%
%The concept of CBDCs, or national digital currencies ? the scenario in which the trend of digital currencies gets adopted by a federal regulator, essentially under its rules, with the central bank issuing digital fiat money, rather than cryptocurrencies in their most popular, decentralized form and becomes not only a regulator, but clients? account holder as well ? has attracted many governments across the world. Some of them have already implemented the idea, some keep researching, while others ? like Germany ? have dismissed the idea altogether. Here?s a list of those countries along with their reasonings for/against CBDCs.
%
%
%For the most avid supporters of digital currencies like Bitcoin, the idea of having a central banking system goes against everything they believe. After all, Bitcoin was created as a means for people to ditch the banking system and allow them the freedom to handle their own financial affairs.
%
%The most pressing reason for a government to launch a cryptocurrency is the success of private cryptocurrencies. This isn?t out of some misplaced sense of competition, but rather because governments would be loathe to lose control of their nation?s monetary policy. When asked about the potential governmental response to cryptocurrencies, former IMF Chief Economist Kenneth Rogoff?s response was, ?When it comes to the monetary system, the government makes the rules. You cannot win the game. If they're not winning, they will change the rules.? If this doesn?t mean stringent regulations?and governments so far have been much more concerned about the scammy or bubbly elements of the crypto world than its capacity for toppling fiat?it could mean govtcoins.
%
%
%Stablecoins carry several advantages over govtcoins. For one, since stablecoins are not ?issued? by nation states, users do not have to worry about government surveillance. In countries helmed by oppressive governments, decentralized stablecoins could still be popular for this reason.
%
%
% Additionally, it?s reasonable to project that if governments can program their currencies, taxes will be built into the govtcoin. If people or entities wish to avoid taxes, they could choose to stash their capital in decentralized stablecoins over currency havens or other govtcoins (though this is obviously not recommended - ?always pay your doctor and the IRS?).
%
%
%Finally, stablecoins also offer potentially easier access as governments might attempt to restrict ownership of their cryptocurrencies to own-country citizens.



%https://www.bis.org/cpmi/publ/d174.pdf
%https://www.bankofcanada.ca/wp-content/uploads/2017/11/sdp2017-16.pdf


\subsection{Promoting Trust in the Crypto Space}
Discussion on Paxos, Gemini vs Tether. How having monthly attestations increase the reliability of Paxos and Gemini?

For a stablecoin to consistently trade at par, its issuer must, in effect, convince a wide a community of future tokenholders of its future solvency. And that degree of trust can be hard to maintain, as it involves the psychology of the market, which can shift significantly over time.
Consider Tether's predicament. Whether it has the funds it says it has isn't the only question. The other, perhaps even more important, is whether the market believes those commitments will hold up against a wider environment of waning confidence.

For reserves-backed stablecoins, this includes practices such as: naming the banking relationship so that users can properly assess the underlying counterparty risk; committing to independent security audits of the underlying code to show that tokens are destroyed when funds are redeemed; holding regular attestations of the firms' balances by trusted third-party auditors. (Note: this does not mean a full "audit" per se. Calls for an audit of Tether were misleading; there is no way that a crypto system's past transactions can be audited in the traditional sense. Instead, proofs rely on attestations as to the accuracy of the firm's claims about its balances at a point in time.)

(https://www.coindesk.com/the-delicate-psychology-of-stablecoins/)

%to be completed >> @Mahsa

%their value is designed not to fluctuate
%
%
%
%Trust: In addition to fostering trust in cryptocurrencies, the other main aim of the Gemini dollar is to promote the use of coins that operate on blockchain. Right now, the main reason people put money in cryptocurrencies is to make more money. ?When something is a store of value, you generally don?t want to spend it,? Tyler Winklevoss told Business Insider. With stablecoins, there?s no money-making incentive. People who buy them will do so presumably for the unique advantages transacting with cryptocurrencies provides, like seamlessly (and fee-lessly) sending money overseas.
%
%
%
%https://medium.com/@usdx/five-issues-confronting-stablecoins-for-the-rest-of-2018-75245820fdb5

\subsection{Smart Insurance}
combine programmability + stability > when the flight gets cancelled or delayed the smart contract will \textbf{automatically} pay the passengers. In this situation, an stable coin is needed.
from the blockchain report: In travel and many other smart insurance use cases, it would be preferable to denominate the smart contract with a stablecoin rather than a more volatile cryptocurrency, such as ether (ETH). Generally, people take out insurance to reduce risk and would therefore want smart contract insurance underpinned by a stable currency.

\textit{Other possible solutions to lending:} ~\textblue{While stable coin is one of them what can be the other possible solutions?}

\textblue{Taken from ~\cite{okoyetoward}, paraphrasing needed!}
Addressing monetary instability:
\begin{itemize}
	\item The rate of release of new currency into the system could be modified to en-
	able new currency to be introduced at (i) a more insightful rate or (ii) based
	on some internal metrics of the system like number of transactions. [Remark:
	an insightful rate has been elusive despite many alt-coins customizing the
	schedule and it is difficult to see how metrics could not be gamed].
	\item A cryptocurrency can also use explicit pegging but it is no better suited to this system than standard currencies.
	\item A central bank could manage currency circulation while allowing other as-
	pects to be decentralized. [	Remark:Central banks have been historically
	unsuccessful at using money circulation as a target].
	\item The loan could be use the cryptocurrency as the medium of exchange but
	use a stable (e.g.,	government) currency as the unit of account.
\end{itemize}

\textit{Other possible solutions to lending:} While stable coin is one of them what can be the other possible solutions?

\section{Comparison Framework} % Put the code for the framework
Define the properties that are considered during the design of stable coins.
~\textblue{Collateralization info from ~\cite{bitmex}, decide which projects to choose that exemplify each category best.}
\begin{table}[]
	\begin{tabular}{|l|l|l|l|l|}
		\hline
		& Collateralization (+ the value of the collateral) & Price Oracle & Centralization  \\ \hline
		BitShares (BitUSD) &  Crypto-collateralized &  No & \\ \hline
		BitBay & Non-collateralized & &  \\ \hline
		 DAI& Crypto-collateralized (ETH)  &  Yes& No \\ \hline
		 BitShares&Crypto-collateralized&Yes&\\ \hline
		 Basis&Non-collateralized&& \\ \hline
		 Tether & Fiat-collateralized (USD) & & Yes \\ \hline
		  &&& \\ \hline
	\end{tabular}
\end{table}

Decentral price oracle and Schelling point~\cite{cryptoinsider}
DAI and Bitshares~\cite{cryptoinsider}

% Tether, the largest stablecoin in terms of its market value of approximately $2.7 billion USD, illustrates the underlying demand for a stablecoin
\section{Critical Issues with Stable Coins}

--Most stable coins fail (this is historically true, it doesn?t speak to the current stable coins). When they fail, they may or may not be redeemable for the promised amount. That is the harsh truth. None of the old stable coins are with us today aside from Nubits?. and Nubits has recently been trading for under .50 cents here in late March 2018. TUSD, Tether, and Dia all seem solid in there here and now, and they likely have learned from the mistakes of their formers, but if we are being honest about the history of stable coins, it is as volatile as the history of crypto itself.

--Some say that stable coins bring stability to the crypto space, but that idea doesn?t seem to be empirically proven. Instead, I?d argue that the perks of stable coins are in the ability for traders to go to a dollar quickly and the ability of exchanges to increase liquidity. Further, I?d argue that stable coins actually add volatility to the crypto space, because they allow for more speculation (which in theory creates stability, but which in practice hasn?t really seemed to do anything of the sort). The bottom line on this point is that the crypto space isn?t a stable place, and no type of cryptocurrency, regulation, or derivative I?ve seen has yet to change that (so here the point is, let us not give credit where it is not due). The term used is ?stable coin? or ?price-stable cryptocurrency,? not because this coin brings stability, but because the token ideally holds a stable value.

--Stable coins generally require us to trust a central third party. Each stable coin has its own way around being overly centralized, but there generally needs to be some way to manage these assets to ensure their stability (even though they tend to be blockchain based). Anything centralized requires the trust of a third party to some extent. Requiring the trust of a third party in crypto goes against the concept of crypto somewhat. This isn?t a problem per-say, but it is worth noting. With that said, most exchanges aren?t decentralized and thus using a stable coin as a currency on an already centralized exchange is hardly the a reason not to use stable coins (or exchanges).

-Discussion about collateralization: The second type of stable coin is partly collateralized. In this case, the platform holds dollars equal to, say 50\%, of the value of the coins in circulation.The problem with this variant will be familiar to any monetary policy maker whose central bank has sought to peg an exchange rate while holding reserves that are only a fraction of its liabilities.
If some coin owners harbor doubts about the durability of the peg, they will sell their holdings. The platform will have to purchase them using its dollar reserves to keep their price from falling. But, because the stock of dollar reserves is limited, other investors will scramble to get out before the cupboard is bare. The result will be the equivalent of a bank run, leading to the collapse of the peg.

\section{Discussion}
%%https://gemini.com/wp-content/themes/gemini/assets/img/dollar/gemini-dollar-whitepaper.pdf -> ref

%Mahsa:
%Other stability benchmarks besides the US dollar are also being employed by stablecoins, including baskets of various fiat currencies (e.g., IMF Special Drawing Rights), commodities or other tangible assets (e.g., gold, real estate), or economic measures (e.g., indexed inflation).
%stable coins have inbuilt mechanism to minimize exchange rate volatility.
%Another important point to emphasize is that stablecoins are simply price-stabilized cryptocurrencies, meaning they incorporate many of bitcoin or ether?s most compelling features: programmability (e.g., smart contract integration), efficiency (e.g., low-to-zero transaction fees, fast settlement times), fungibility, open (i.e., permissionless) access, and so on.
%Market makers and traders may also welcome the steadier nature of stablecoins as they carry out their daily operations.
