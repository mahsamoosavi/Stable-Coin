% !TEX root = ../main.tex
\section{Introduction}
Very quick introduction to blockchain/crypto
What is stability vs volatility.
About coin supply and volatility~\cite{sams2015note}.
Short-term stability is important for transactions and long-term stability is important for holding`\cite{forbes}.


\section{The current state of the stable coins} % Didem :

%different ways and issues with each.
Stablecoins can be categorized into three groups based on the way they achieve stability: fiat-collateralized, crypto-collateralized, and non-collateralized.

~\textbf{1) Fiat-collateralized stablecoins:} These type of stable coins are backed by fiat currency and backing by USD is one of the most common types. Generally, there is a 1:1 peg between the fiat currency and the stablecoin that indicates a convergence between their values~\cite{linkedin}. USD being one of the most common choices for the fiat currecncy to back the stablecoin, IBM states that they are also interested in projects that use other national fiat currencies, as they will be helpful for IBM's blockchain integration~\cite{cointelegraph}.

Tether and TrueUSD are USD are prominent examples of USD pegged tokens. Some projects like Digix Gold Token prefer to use gold to back their stablecoin, as gold has a relatively slow increase in its value compared to fiat currencies. %Reference??

~\textit{Discussion about centralization:} Backing up with fiat currency means that there is a need for third party. The amount of money to back the stablecoin up should be held in an account~\cite{techrev}. The involvement of a third party causes controversy in the community, as the third party can just deny giving money to the users. Tether explains this point as follows~\cite{cryptoinsider}:

\begin{quote}
"Redemptions will not be unreasonably denied, but we reserve the right to selectively deny redemption and creation of Tethers on a case-by-case basis."
\end{quote}

%Globcoin: less reliance on US dollars but still centralized.~\cite{cryptoinsider}

~\textbf{2) Crypto-collateralized stablecoins:} These type of stablecoins uses other cryptocurrencies as a back up value rather than a fiat currency. Overcollateralization is needed this case as the underlying cryptocurrency is also volatile~\cite{linkedin}. MakerDAO and Reserve use this approach. Reserve utilizes a smart contract to back the stablecoin with another cryptocurrency~\cite{cointelegraph}.

~\textbf{3) Non-collateralized stablecoins:} Unlike the previous types of stable coins, these aren't back by fiat currencies of another cryptocurrency. The stability is achieved algorithmically~\cite{linkedin}. Basis is one of the first projects that use this approach.

Basis and Carbon use the dual-token model~\cite{cryptoinsider}. There is dynamic adjustment of the existing supply of the stablecoin. While one token is stable, the other is used to achieve the stability of the value.

%The dual-token model — stablecoins and shares One token is stable, and the other is used to whip it back onto the narrow track. Roughly, the idea is that supply of the stable token is dynamically increased and decreased. ex: Basis, Carbon. Then the dual-token model has two flavours: the kind where the stable token is created by locking up other crypto (e.g. DAI and Havven), and the kind where it is not (e.g. Basis, Carbon). ~\cite{cryptoinsider}  DAI Havven crypto collat?

Another approach is the seigniorage shares method\cite{overview}. Here, the smart contract automatically adjusts the supply based on the algorithm to achieve stability in the value.
\section{Problems that stable coin addresses} %Mahsa:
% + their alternative solutions:
\subsection{Government Surveillance}
Stablecoins carry several advantages over govtcoins. For one, since stablecoins are not ?issued? by nation states, users do not have to worry about government surveillance. In countries helmed by oppressive governments, decentralized stablecoins could still be popular for this reason. Additionally, it?s reasonable to project that if governments can program their currencies, taxes will be built into the govtcoin. If people or entities wish to avoid taxes, they could choose to stash their capital in decentralized stablecoins over currency havens or other govtcoins (though this is obviously not recommended - ?always pay your doctor and the IRS?). Finally, stablecoins also offer potentially easier access as governments might attempt to restrict ownership of their cryptocurrencies to own-country citizens.

\subsection{Lending}
~\textblue{Talk about lending and other possible problems that stable coins may solve. }

One of the main challenges observed in lending is  monetary instability~\cite{okoyetoward}. The volatile nature of fiat currencies or the cryptocurrencies pose a risk to both the cash taker and the cash provider. The increase in the value of the currency will cause the cash taker to owe more than the initial amount. On the other hand, if there is a decrease in the value, the cash provider will end up receiving an amount less than he/she lent.



\textit{Other possible solutions to lending:} ~\textblue{While stable coin is one of them what can be the other possible solutions?}

\textblue{Taken from ~\cite{okoyetoward}, paraphrasing needed!}
Addressing monetary instability:
\begin{itemize}
	\item The rate of release of new currency into the system could be modified to en-
	able new currency to be introduced at (i) a more insightful rate or (ii) based
	on some internal metrics of the system like number of transactions. [Remark:
	an insightful rate has been elusive despite many alt-coins customizing the
	schedule and it is difficult to see how metrics could not be gamed].
	\item A cryptocurrency can also use explicit pegging but it is no better suited to this system than standard currencies.
	\item A central bank could manage currency circulation while allowing other as-
	pects to be decentralized. [	Remark:Central banks have been historically
	unsuccessful at using money circulation as a target].
	\item The loan could be use the cryptocurrency as the medium of exchange but
	use a stable (e.g.,	government) currency as the unit of account.
\end{itemize}

\textit{Other possible solutions to lending:} While stable coin is one of them what can be the other possible solutions?

\section{Comparison Framework} % Put the code for the framework
Define the properties that are considered during the design of stable coins.
~\textblue{Collateralization info from ~\cite{bitmex}, decide which projects to choose that exemplify each category best.}
\begin{table}[]
	\begin{tabular}{|l|l|l|l|l|}
		\hline
		& Collateralization (+ the value of the collateral) & Price Oracle & Centralization  \\ \hline
		BitShares (BitUSD) &  Crypto-collateralized &  No & \\ \hline
		BitBay & Non-collateralized & &  \\ \hline
		 DAI& Crypto-collateralized (ETH)  &  Yes& \\ \hline
		 BitShares&Crypto-collateralized&Yes&\\ \hline
		 Basis&Non-collateralized&& \\ \hline
		 Tether & Fiat-collateralized (USD) & & Yes \\ \hline
		  &&& \\ \hline
	\end{tabular}
\end{table}

Decentral price oracle and Schelling point~\cite{cryptoinsider}
DAI and Bitshares~\cite{cryptoinsider}

\section{Discussion}
%%https://gemini.com/wp-content/themes/gemini/assets/img/dollar/gemini-dollar-whitepaper.pdf -> ref


